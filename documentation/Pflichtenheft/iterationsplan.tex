\subsection{Ranking der Usecases}
Zuerst eine Liste der UseCases und deren Nummer:
\newline \newline
\begin{tabular} { |c | l |}
	\hline
	\# & Name des Usecases \\
	1 & Checkout\\
	2 & Checkin - Reservierung\\
	3 & Reservierung aufnahmen - Individualgast\\
	4 & Checkin - Walkin\\
	5 & Checkin - Reisegruppe\\
	6 & Reservierung aufnehmen - Reisebüro\\
	7 & Reservierung aufnehmen - Firmen\\
	8 & Reservierung bestätigen\\
	9 & Aufenthalt verlängern\\
	10 & Reservierung stornieren - Vertragspartner\\
	11 & Reservierung stornieren - Individualgast\\
	12 & Zimmer wechseln\\
	13 & Reservierung ändern\\
	\hline
\end{tabular}

\vspace{1cm}

Hier das Ranking der einzelnen Usecases:
\vspace{1cm}

\begin{tabular} { | l | c c c c c c c c c c c c c | }
	\hline
	Name des Usecases & 1 & 2 & 3 & 4 & 5 & 6 & 7 & 8 & 9 & 10 & 11 & 12 & 13 \\ \hline
	Risiko & 5,25 &	4,25 & 4 & 4,5 & 5,5 & 4,75 & 4,5 & 3 & 4 & 4,75 & 4,75 & 3,5 & 3\\
	Komplexität & 8 & 7 & 7 & 6 & 8 & 5 & 5 & 3 & 7 & 5 & 6 & 5 & 5\\
	Anforderungen & 3 & 3 & 2 & 4 & 5 & 5 & 5 & 2 & 2 & 5 & 4 & 2 & 1\\
	Team Knowhow & 5 & 2 & 2 & 3 & 4 & 4 & 3 & 2 & 2 & 4 & 4 & 2 & 1\\
	Technologie & 5 & 5 & 5 & 5 & 5 & 5 & 5 & 5  & 5 & 5 & 5 & 5 & 5\\
	\hline
	Architekturrelevanz	& 9 & 9 & 8 & 8 & 7 & 7 & 7 & 4 & 3 & 5 & 4 & 3 & 3\\
	Benutzerrelevanz & 10 & 10 & 10 & 8 & 8 & 5 & 5 & 9 & 5 & 2 & 2 & 4 & 4 \\
	\hline \hline
	Gesamt & 6 & 5,8 & 5,5 & 5,1 & 5,1 & 4,2 & 4,1 & 4 & 3 & 3 & 2,7 & 2,6 & 2,5\\
\hline
\end{tabular}



\subsection{Überblick}
Zuerst wird der Usecase Checkin - Reservierung (\ref{UseCase_CheckinReservierung}) implementiert, welcher Stammdaten von Gästen und von Zimmern vorraussetzt.
Aus diesen Reservierungen werden dann Aufenthalte, welche in der Timebox 2 mit dem Usecase Checkout (\ref{UseCase_Checkout}) beendet werden können.
Außerdem wird in der zweiten Timebox die Möglichkeit geschaffen, neue \Gls{Reservierung}en von Individualpersonen zu erzeugen und außerdem
das Kunden ohne vorherige \Gls{Reservierung} einfach ins Hotel kommen können und ein freies Zimmer zu beziehen.
Schlussendlich wird nach der dritten Timebox die Bestätigung von Reservierungen mit dem System möglich sein.

\subsection{Timebox 1}
In der ersten Timebox beginnen wir mit der Implementierung des Usecases Checkin - Reservierung (\ref{UseCase_CheckinReservierung}) inklusive der benötigten Architektur,
um eine vorhandene Reservierung zu finden und diese zu Bestätigen.

\subsubsection{Usecases}
In der ersten Timebox wird das Main Success Szenario des Usecases Checkin - Reservierung (\ref{UseCase_CheckinReservierung}) implementiert,
inklusive der folgenden Extensions:

\begin{itemize}
	\item Der \Gls{Kunde} kennt ihre \Gls{Reservierungsnummer} nicht
	\item Die \Gls{Reservierungsnummer} existiert nicht
	\item Die Daten der \Gls{Reservierung} sind nicht vollständig
\end{itemize}
\subsubsection{Architektur}
Die folgende Architektur wird in der ersten Timebox realisiert:

\begin{itemize}
	\item Das komplette Datenbank-Schema
	\item Folgende Oberflächen:
	\begin{itemize}
		\item Reservierungen im System suchen
		\item Ankunft eines \Gls{Kunde}n eintragen
		\item Ändern der Daten eines \Gls{Gast}es
	\end{itemize}
\end{itemize}
\subsubsection{Deliverables}

\subsubsection{Abhängigkeiten}
Gegeben müssen folgende Dinge sein:

\begin{itemize}
	\item Die Stammdaten der Zimmer sind im System vorhanden.
	\item Bestätigte Reservierungen sind bereits im System vorhanden.
\end{itemize}

\subsection{Timebox 2}
In der zweiten Timebox werden weiter Teile des Systems implementiert.
Nach Ende der Timebox 2 kann das System die wichtigsten Szenarien im Hotelbetrieb übernehmen.

\subsubsection{Usecases}
In der Timebox 2 werden die Main Success Szenarios der folgenden Usecases implementiert:

\begin{itemize}
	\item Checkout (\ref{UseCase_Checkout}) inklusive folgender Extension:
	\begin{itemize}
		\item Der \Gls{Gast} kennt seine \Gls{Zimmernummer} nicht.
	\end{itemize}
	\item Reservierung aufnehmen - Individualgast (\ref{UseCase_ReservierungAufnehmenIndividualgast})
	\item Checkin - Walkin (\ref{UseCase_CheckinWalkin})
\end{itemize}

\subsubsection{Architektur}
Folgende Dinge werden im Zuge der Ausarbeitung implementiert:

\begin{itemize}
	\item Code, um eine \Gls{Reservierung}svorschau zu erstellen
	\item Folgende Oberflächen:
	\begin{itemize}
		\item Stammdaten eines neuen \Gls{Gast}es in das System einpflegen.
		\item Eine \Gls{Reservierung}svorschau anzeigen lassen.
		\item Eine neue \Gls{Reservierung} erstellen.
		\item Möglichkeit zur Erstellung von Zwischenrechnungen bzw. \Gls{Rechnung}.
	\end{itemize}
\end{itemize}

\subsubsection{Deliverables}
\subsubsection{Abhängigkeiten}
Um mit der Timebox 2 zu beginnen, muss zumindest das Main Success Szenario des Usecases Checkin - Reservierung (\ref{UseCase_CheckinReservierung}) aus der ersten Timebox funktionstüchtig sein
und die selben Abhängigkeiten wie in der ersten Timebox erfüllt sein.


\subsection{Timebox 3}
In der dritten Timebox werden schlussendlich noch 2 Usecases implementiert, sodass für Individualgäste der gesamte Ablauf mit dem System gemacht werden kann.

\subsubsection{Usecases}
In der Timebox 3 werden die Main Success Szenarios der folgenden Usecases implementiert:

\begin{itemize}
	\item \Gls{Reservierung} bestätigen
	\item \Gls{Reservierung} Stornieren - Individualperson inklusive folgender Extensions:
	\begin{itemize}
		\item Der \Gls{Gast} kennt seine \Gls{Reservierungsnummer} nicht.
		\item Der \Gls{Gast} will die Reservierung zwischen 28 und 15 Tagen vor Ankunft die Reservierung stornieren.
		\item Der \Gls{Gast} will die Reservierung innerhalb von 15 Tagen vor Ankunft die Reservierung stornieren.
	\end{itemize}
\end{itemize}
\subsubsection{Architektur}
\begin{itemize}
	\item Code, welcher die bereits Angefallenen Kosten der Reservierung berechnet.
	\item Folgende Oberflächen:
	\begin{itemize}
		\item Reservierung bestätigen
		\item Eine Reservierung stornieren
	\end{itemize}
\end{itemize}
\subsubsection{Deliverables}
\subsubsection{Abhängigkeiten}
Damit die dritte Timebox realisiert werden kann, muss der Usecase Reservierung aufnehmen - Individualgast (\ref{UseCase_ReservierungAufnehmenIndividualgast}) aus der zweiten Timebox funktionieren.
