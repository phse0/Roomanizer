\subsection{Ranking der Usecases}

\begin{tabular} { l c c c c c c c c }

	Name des Usecases & Risiko & Komplexität & Anforderungen & Team Knowhow & Technologie& Architekturrelevanz & Benutzerrelevanz & Gesamt \\
	Checkout & 5,25 & 8 & 3 & 5 & 5 & 9 & 10 & 6,0625 \\
	Checkin - Reservierung & 4,25 & 7 & 3 & 2 & 5 & 9 & 10 & 5,8125 \\
	Reservierung aufnahmen - Individualgast & 4 & 7 & 2 & 2 & 5 & 8 & 10 & 5,5 \\
	Checkin - Walkin & 4,5 & 6 & 4 & 3 & 5 & 8 & 8 & 5,125 \\
	Checkin - Reisegruppe & 5,5 & 8 & 5 & 4 & 5 & 7 & 8 & 5,125 \\
	Reservierung aufnehmen - Reisebüro & 4,75 & 5 & 5 & 4 & 5 & 7 & 5 & 4,1875 \\
	Reservierung aufnehmen - Firmen & 4,5 & 5 & 5 & 3 & 5 & 7 & 5 & 4,125 \\
	Reservierung bestätigen & 3 & 3 & 2 & 2 & 5 & 4 & 9 & 4 \\
	Aufenthalt verlängern & 4 & 7 & 2 & 2 & 5 & 3 & 5 & 3 \\
	Reservierung stornieren - Vertragspartner & 4,75 & 5 & 5 & 4 & 5 & 5 & 2 & 2,9375 \\
	Reservierung stornieren - Individualgast & 4,75 & 6 & 4 & 4 & 5 & 4 & 2 & 2,6875 \\
	Zimmer wechseln & 3,5 & 5 & 2 & 2 & 5 & 3 & 4 & 2,625 \\
	Reservierung ändern & 3 & 5 & 1 & 1 & 5 & 3 & 4 & 2,5 \\
	
\end{tabular}



\subsection{Überblick}
Zuerst wird der Usecase \textbf{Checkin - Reservierung} implementiert, welcher Stammdaten von Gästen und von Zimmern vorraussetzt.
Aus diesen Reservierungen werden dann Aufenthalte, welche in der Timebox 2 mit dem Usecase \textbf{Checkout} beendet werden können.
Außerdem wird in der zweiten Timebox die Möglichkeit geschaffen, neue \Gls{Reservierung}en von Individualpersonen zu erzeugen und außerdem
das Kunden ohne vorherige \Gls{Reservierung} einfach ins Hotel kommen können und ein freies Zimmer zu beziehen.
Schlussendlich wird nach der dritten Timebox die Bestätigung von Reservierungen mit dem System möglich sein.

\subsection{Timebox 1}
In der ersten Timebox beginnen wir mit der Implementierung des Usecases \textbf{Checkin - Reservierung} inklusive der benötigten Architektur,
um eine vorhandene Reservierung zu finden und diese zu Bestätigen.

\subsubsection{Usecases}
In der ersten Timebox wird das Main Success Szenario des Usecases \textbf{Checkin - Reservierung} implementiert,
inklusive der folgenden Extensions:

\begin{itemize}
	\item Der \Gls{Kunde} kennt ihre \Gls{Reservierungsnummer} nicht
	\item Die \Gls{Reservierungsnummer} existiert nicht
	\item Die Daten der \Gls{Reservierung} sind nicht vollständig
\end{itemize}
\subsubsection{Architektur}
Die folgende Architektur wird dafür entwickelt:

\begin{itemize}
	\item Ändern der Stammdaten eines Gastes
	\item Eine Reservierung suchen
	\item Vorschlag von \Gls{Zimmer}n in der Kategorie, welche in der \Gls{Reservierung} steht.
\end{itemize}
\subsubsection{Deliverables}
\subsubsection{Abhängigkeiten}
Gegeben müssen folgende Dinge sein:

\begin{itemize}
	\item Die Stammdaten der Zimmer sind im System vorhanden.
	\item Bestätigte Reservierungen sind bereits im System vorhanden.
\end{itemize}

\subsection{Timebox 2}
In der zweiten Timebox werden weiter Teile des Systems implementiert.
Nach Ende der Timebox 2 kann das System die wichtigsten Szenarien im Hotelbetrieb übernehmen.

\subsubsection{Usecases}
In der Timebox 2 werden die Main Success Szenarios der folgenden Usecases implementiert:

\begin{itemize}
	\item Checkout inklusive folgender Extension:
	\begin{itemize}
		\item Der \Gls{Gast} kennt seine Gls{Zimmernummer} nicht.
	\end{itemize}
	\item Reservierung aufnehmen - Individualgast
	\item Checkin - Walkin
\end{itemize}
\subsubsection{Architektur}
Folgende Dinge werden im Zuge der Ausarbeitung implementiert:

\begin{itemize}
	\item Stammdaten eines neuen \Gls{Gast}es in das System einpflegen.
	\item Das System kann Zwischenrechnungen bzw. Rechnung erstellen.
\end{itemize}

\subsubsection{Deliverables}
\subsubsection{Abhängigkeiten}
Um mit der Timebox 2 zu beginnen, muss zumindest das Main Success Szenario des Usecases \textbf{Checkin - Reservierung} aus der ersten Timebox funktionstüchtig sein
und die selben Abhängigkeiten wie in der ersten Timebox erfüllt sein.


\subsection{Timebox 3}
In der dritten Timebox werden schlussendlich noch 2 Usecases implementiert, sodass für Individualgäste der gesamte Ablauf mit dem System gemacht werden kann.

\subsubsection{Usecases}
In der Timebox 3 werden die Main Success Szenarios der folgenden Usecases implementiert:

\begin{itemize}
	\item Reservierung bestätigen
	\item Reservierung Stornieren - Individualperson inklusive folgender Extensions:
	\begin{itemize}
		\item Der \Gls{Gast} kennt seine \Gls{Reservierungsnummer} nicht.
		\item Der \Gls{Gast} will die Reservierung zwischen 28 und 15 Tagen vor Ankunft die Reservierung stornieren.
		\item Der \Gls{Gast} will die Reservierung innerhalb von 15 Tagen vor Ankunft die Reservierung stornieren.
	\end{itemize}
\end{itemize}
\subsubsection{Architektur}
Im System wird eine Möglichkeit hinzugefügt, um eine \Gls{Reservierung} bestätigen bzw stornieren zu können.
\subsubsection{Deliverables}
\subsubsection{Abhängigkeiten}
Damit die dritte Timebox realisiert werden kann, muss der Usecase \textbf{Reservierung aufnehmen - Individualgast} aus der zweiten Timebox funktionieren.
