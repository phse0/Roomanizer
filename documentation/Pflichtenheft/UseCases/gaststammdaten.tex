\subsubsection{Gaststammdaten verwalten}

\paragraph{Kurzbeschreibung}
Der Mitarbeiter sucht im System mit Vor- und/oder Nachname nach einem Gast. Das System listet ihm alle Suchergebnisse, mit weiteren Informationen versehen, auf. Der Mitarbeiter wählt den zu verwaltenden Gast aus. Der Mitarbeiter kann vorhandene Daten zum Gast nun verändern oder neue Daten anlegen. Zum Schluss bestätigt der Mitarbeiter die gemachten Änderungen.

\paragraph{Stakeholders und Akteure}
\begin{itemize}
	\item Backoffice-Mitarbeiter - Benötigt Gaststammdaten für Statistiken usw.
	\item \Gls{Rezeptionist} - Braucht Gaststammdaten für fast alle seine Interaktionen mit dem System
	\item \Glspl{Gast} - Reibungsloser Ablauf und schnelle Bearbeitung seiner Wünsche indem er schon im System hinterlegt ist bzw. erst eingefügt wird.
\end{itemize}

\paragraph{Vorbedingung}


\paragraph{Nachbedingung}

\paragraph{Basisablauf}
\begin{enumerate}
	\item Mitarbeiter öffnet Gastsuche im System
	\item Der Mitarbeiter gibt vorhandene Daten des Gastes ein
	\item Das System zeigt Gäste an, die unter die Suchkriterien fallen
	\item Der Mitarbeiter wählt gewünschten Gast aus
	\item Der Mitarbeiter ändert die Daten des Gastes
	\item Der Mitarbeiter bestätigt seine Eingaben
\end{enumerate}

\paragraph{Alternativer Ablauf}
\begin{longenum}
	\item
	\item
	\item
	\item
	\item
	\begin{longenum}
		\item Der Mitarbeiter will den Gast aus dem System löschen
		\begin{longenum}
			\item Der Mitarbeiter markiert den Gast als zu löschen
			\item \emph{weiter mit Basisablauf Punkt 6}
		\end{longenum}
	\end{longenum}
\end{longenum}

\paragraph{Besondere Anforderungen}

\paragraph{Technologie und Daten Variationsliste}

\paragraph{Benutzerfrequenz}
Selten - 1 mal pro Woche

\paragraph{Offene Punkte}

\newpage
