\subsubsection{Gaststammdaten verwalten}

\paragraph{Kurzbeschreibung}
Der \Gls{Mitarbeiter} sucht im System mit Vor- und/oder Nachname nach einem \Gls{Gast}. Das System listet ihm alle Suchergebnisse, mit weiteren Informationen versehen, auf. Der \Gls{Mitarbeiter} wählt den zu verwaltenden \Gls{Gast} aus. Der \Gls{Mitarbeiter} kann vorhandene Daten zum \Gls{Gast} nun verändern oder neue Daten anlegen. Zum Schluss bestätigt der \Gls{Mitarbeiter} die gemachten Änderungen.

\paragraph{Stakeholders und Akteure}
\begin{itemize}
	\item \Gls{Backoffice}-\Gls{Mitarbeiter} - Benötigt Gaststammdaten für Statistiken usw.
	\item \Gls{Rezeptionist} - Braucht Gaststammdaten für fast alle seine Interaktionen mit dem System
	\item \Glspl{Gast} - Reibungsloser Ablauf und schnelle Bearbeitung seiner Wünsche indem er schon im System hinterlegt ist bzw. erst eingefügt wird.
\end{itemize}

\paragraph{Vorbedingung}
\begin{itemize}
	\item keine
\end{itemize}

\paragraph{Nachbedingung}
\begin{itemize}
	\item keine
\end{itemize}

\paragraph{Basisablauf}
\begin{enumerate}
	\item \Gls{Mitarbeiter} öffnet Gastsuche im System
	\item Der \Gls{Mitarbeiter} gibt vorhandene Daten des Gastes ein
	\item Das System zeigt \Glspl{Gast} an, die unter die Suchkriterien fallen
	\item Der \Gls{Mitarbeiter} wählt gewünschten \Gls{Gast} aus
	\item Der \Gls{Mitarbeiter} ändert die Daten des \Gls{Gast}es
	\item Der \Gls{Mitarbeiter} bestätigt seine Eingaben
	\item Das System speichert die Änderungen
\end{enumerate}

\paragraph{Alternativer Ablauf}
\begin{longenum}
	\item
	\item
	\item
	\item
	\item
	\begin{longenum}
		\item Der \Gls{Mitarbeiter} will den \Gls{Gast} aus dem System löschen
		\begin{longenum}
			\item Der Mitarbeiter markiert den Gast als zu löschen
			\item \emph{weiter mit Basisablauf Punkt 6}
		\end{longenum}
	\end{longenum}
\end{longenum}

\paragraph{Besondere Anforderungen}
\begin{itemize}
	\item keine
\end{itemize}

\paragraph{Benutzerfrequenz}
Selten - 1 mal pro Woche

\paragraph{Offene Punkte}

\newpage
