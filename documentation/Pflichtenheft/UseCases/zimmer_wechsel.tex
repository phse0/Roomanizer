\subsubsection{Zimmer wechseln}
\label{UseCase_ZimmerWechseln}

\paragraph{Kurzbeschreibung}
Ein \Gls{Gast} äußert an der \Gls{Rezeption} den Wunsch, das \Gls{Zimmer} zu wechseln. Der \Gls{Rezeptionist} gibt die \Gls{Zimmernummer} in das System ein, und kann dem \Gls{Gast} ein neues \Gls{Zimmer} zuweisen. Dabei werden alle bisher angefallenen Kosten auf das neue \Gls{Zimmer} übertragen, und sind nur noch auf diesem aufrubar.

\paragraph{Stakeholders und Akteure}
\begin{itemize}
	\item Rezeptionist - Schnelle Zuweisung des neuen \Gls{Zimmer}s an den \Gls{Gast}
	\item Gast - Möglichst schneller Bezug des neuen \Gls{Zimmer}s
	\item Manager - Grund aus Kundensicht für den \Gls{Zimmer}wechsel
\end{itemize}

\paragraph{Vorbedingung}
\begin{itemize}
	\item Der \Gls{Gast} muss bereits ein \Gls{Zimmer} bezogen haben.
\end{itemize}

\paragraph{Nachbedingung}
\begin{itemize}
	\item Dem \Gls{Gast} muss ein neues Zimmer zugeteilt sein.
	\item Die bereits bestehenden Rechnungen des \Gls{Gast}es dürfen nur noch auf dem neuen \Gls{Zimmer} aufrufbar sein.
\end{itemize}

\paragraph{Basisablauf}
\begin{longenum}
	\item Der \Gls{Gast} kommt zur \Gls{Rezeption} und gibt seine \Gls{Zimmernummer} und eventuell auch \Gls{Belegungsnummer} an.
	\item Der \Gls{Rezeptionist} sucht den Gast anhand der \Gls{Zimmernummer}.
	\item Das System zeigt Informationen zum \Gls{Gast} und \Gls{Zimmer} an.
	\item Der \Gls{Rezeptionist} teilt dem System mit, dass der \Gls{Gast} das \Gls{Zimmer} wechseln will.
	\item Das System schlägt dem \Gls{Rezeptionist} \Gls{Zimmer} der gleichen Kategorie, die für die restliche Verweildauer des \Gls{Gast}es vefügbar sind, zum Wechsel vor.
	\item Der \Gls{Rezeptionist} wählt eines dieser \Gls{Zimmer} für den Wechsel aus.
	\item Der \Gls{Gast} übergibt dem \Gls{Rezeptionist} die Schlüssel für das alte
	\Gls{Zimmer}, und erhält im Gegenzug die des neuen \Gls{Zimmer}s.
\end{longenum}

\paragraph{Alternativer Ablauf}
\begin{longenum}
	\item
	\item
	\begin{longenum}
		\item Der \Gls{Gast} kennt seine \Gls{Zimmernummer} nicht.
		\begin{longenum}
			\item Der \Gls{Gast} gibt stattdessen seinen Namen an.
			\item Der \Gls{Rezeptionist} sucht den \Gls{Gast} anhand des Namens.
			\item \emph{weiter mit Basisablauf Punkt 3}
		\end{longenum}
	\end{longenum}
	\item
	\item
	\item
	\begin{longenum}
		\item Es sind keine \Gls{Zimmer} in der selben Kategorie frei:
		\begin{longenum}
			\item Das System gibt alle freien \Gls{Zimmer}, die für die restliche Verweildauer des \Gls{Gast}es verfügbar sind, zusammen mit einer Warnung aus.
			\item Der \Gls{Rezeptionist} wählt eines dieser \Gls{Zimmer} aus, und verhandelt falls notwendig mit dem \Gls{Gast} neue Konditionen aus.
			\item \emph{weiter mit Basisablauf Punkt 7}
		\end{longenum}
		\item Es sind überhaupt keine \Gls{Zimmer} mehr frei:
		\begin{longenum}
			\item \emph{UseCase abbrechen}
		\end{longenum}
	\end{longenum}
	\item
	\item
\end{longenum}

\paragraph{Besondere Anforderungen}
\begin{itemize}
	\item keine
\end{itemize}

\paragraph{Benutzerfrequenz}
selten

\newpage
