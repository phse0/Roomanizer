\subsubsection{Checkin - Reisegruppe}
\label{UseCase_CheckinReisegruppe}

\paragraph{Kurzbeschreibung}
Der \Gls{Reiseleiter} kommt zur \Gls{Rezeption} und meldet die Ankunft seiner Reisegruppe. Die \Gls{Reservierung} liegt im System unter dem Namen des \Gls{Reiseleiter}s vor. Nun kann der \Gls{Reiseleiter} die Daten aller Reisegruppenmitglieder angeben, damit sie erfasst werden können.
Dann werden die \Gls{Zimmer} auf die Mitglieder der Reisegruppe aufgeteilt. Der \Gls{Rezeptionist} verteilt noch die Schlüssel und eventuell geforderte Informationen zu Hotel und Umgebung.

\paragraph{Stakeholders und Akteure}
\begin{itemize}
	\item \Gls{Rezeptionist} - Schnelle und zuverlässige Bearbeitung des \Gls{Checkin}s, Konzentration auf den  \Gls{Reiseleiter}, nicht auf den Bildschirm
	\item \Gls{Reiseleiter} - Reibungsloser Ablauf und schnelle Bearbeitung, um die Mitglieder der Reisegruppe nicht zu lange warten zu lassen
	\item \Glspl{Gast} - Reibungsloser Ablauf und schnelle Bearbeitung, schneller Bezug der \Gls{Zimmer}
\end{itemize}

\paragraph{Vorbedingung}
\begin{itemize}
	\item Allgemeine Vorbedingungen
\end{itemize}

\paragraph{Nachbedingung}
\begin{itemize}
	\item Alle der Reisegruppe zugeordneten \Gls{Zimmer} wurden erfolgreich den \Glspl{Gast}n zugeteilt.
\end{itemize}

\paragraph{Basisablauf}
\begin{enumerate}
	\item Ein \Gls{Reiseleiter} kommt zur \Gls{Rezeption} und gibt seine \Gls{Reservierungsnummer} an.
	\item Der \Gls{Rezeptionist} sucht im System anhand der \Gls{Reservierungsnummer} die \Gls{Reservierung} des \Gls{Gast}es.
	\item Das System zeigt die Reservierungsinformationen an.
	\item Der \Gls{Rezeptionist} überprüft die Vollständigkeit der Reservierungsdaten im System.
	\item Der \Gls{Rezeptionist} trägt noch die gewünschten \Gls{Zusatzleistung}en der Reisegruppe ein.
	\item Der \Gls{Rezeptionist} teilt dem System mit, dass dem \Gls{Gast} das \Gls{Zimmer} übergeben wird.
	\item Der \Gls{Rezeptionist} übergibt dem Reiseleiter die Schlüssel und weitere Informationen zu Hotel und Umgebung.
\end{enumerate}

\paragraph{Alternativer Ablauf}
\begin{longenum}
	\item
	\begin{longenum}
		\item Der \Gls{Reiseleiter} kennt die Reservierungsnummer nicht:
		\begin{longenum}
			\item Der Reiseleiter gibt stattdessen seinen Namen an:
			\item Der \Gls{Rezeptionist} sucht die \Gls{Reservierung} im System anhand des Namens und des Datums.
			\begin{longenum}
				\item Das System findet keine \Gls{Reservierung} mit dem angegebenem Namen:
				\begin{longenum}
					\item \emph{UseCase abbrechen}
				\end{longenum}
			\end{longenum}
			\item \emph{weiter mit Basisablauf Punkt 3}
		\end{longenum}
	\end{longenum}
	\item
	\begin{longenum}
		\item Es existiert keine \Gls{Reservierung} mit der gegebenen \Gls{Reservierungsnummer}:
		\item Der \Gls{Rezeptionist} sucht die \Gls{Reservierung} im System anhand des Namens.
		\begin{longenum}
			\item Das System findet keine \Gls{Reservierung} mit dem angegebenem Namen:
			\begin{longenum}
				\item \emph{UseCase abbrechen}
			\end{longenum}
		\end{longenum}
		\item \emph{weiter mit Basisablauf Punkt 3}
	\end{longenum}
	\item
	\item
	\begin{longenum}
		\item Die Daten der \Gls{Reservierung} sind nicht vollständig:
		\begin{longenum}
			\item Der \Gls{Rezeptionist} fragt den \Gls{Reiseleiter} nach den restlichen Daten, und trägt diese im System ein.
			\item \emph{weiter mit Basisablauf Punkt 5}
		\end{longenum}
	\end{longenum}
	\item
	\item
	\item
\end{longenum}

\paragraph{Besondere Anforderungen}
\begin{itemize}
	\item keine
\end{itemize}


\paragraph{Benutzerfrequenz}
mehrmals wöchentlich (relativ häufig)

\paragraph{Offene Punkte}
\begin{itemize}
	\item Umgang mit Überbuchungen
	\item Ungereinigte Zimmer
\end{itemize}

\newpage

