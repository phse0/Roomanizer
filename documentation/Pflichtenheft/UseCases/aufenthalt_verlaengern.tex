\subsubsection{Aufenthalt verlängern}
\label{UseCase_AufenthaltVerlängern}

\paragraph{Kurzbeschreibung}
Ein \Gls{Gast} kommt zur \Gls{Rezeption} und möchte seinen \Gls{Aufenthalt} verlängern. Nachdem der \Gls{Gast} seine Daten dem \Gls{Rezeptionist}en mitgeteilt hat, sucht dieser das entsprechende Zimmer. Anschließend überprüft der \Gls{Rezeptionist}, anhand der \Gls{Belegungsvorschau} ob eine Verlängerung für den gewünschten Zeitraum möglich ist. Anschließend wird das neue Abreisedatum eingetragen.

\paragraph{Stakeholders und Akteure}
\begin{itemize}
\item \Gls{Rezeptionist} - Schnelle und zuverlässige Verlängerung des \Gls{Aufenthalt}es eines \Gls{Gast}es
\item \Glspl{Gast} - Reibungsloser Ablauf und schnelle Bearbeitung von Verlängerungen des \Gls{Aufenthalt}es
\end{itemize}

\paragraph{Vorbedingung}
\begin{itemize}
\item Der \Gls{Gast} muss bereits ein \Gls{Zimmer} bezogen haben.
\end{itemize}

\paragraph{Nachbedingung}
\begin{itemize}
	\item Der \Gls{Aufenthalt} ist verlängert
\end{itemize}

\paragraph{Basisablauf}
\begin{enumerate}
\item Der \Gls{Gast} kommt zur \Gls{Rezeption} und gibt seine \Gls{Zimmernummer} und eventuell auch \Gls{Belegungsnummer} an.
\item Der \Gls{Rezeptionist} sucht den \Gls{Gast} anhand der \Gls{Zimmernummer}.
\item Das System zeigt Informationen zum \Gls{Gast} und \Gls{Zimmer} an.
\item Der \Gls{Rezeptionist} teilt dem System mit, dass der \Gls{Gast} den \Gls{Aufenthalt} verlängern will.
\item Das System sagt dem \Gls{Rezeptionist}, dass das \Gls{Zimmer} des \Gls{Gast}es bis zum gewünschten Abreisedatum frei ist.
\item Der \Gls{Rezeptionist} bestätigt die Verlängerung des Aufenthaltes des \Gls{Gast}es.
\end{enumerate}

\paragraph{Alternativer Ablauf}
\begin{longenum}
	\item
	\begin{longenum}
		\item Der \Gls{Gast} kennt seine \Gls{Zimmernummer} nicht:
		\begin{longenum}
			\item Der \Gls{Gast} gibt stattdessen seinen Namen an.
			\item Der \Gls{Rezeptionist} sucht den \Gls{Gast} anhand des Namens.
			\item \emph{weiter mit Basisablauf Punkt 3}
		\end{longenum}
	\end{longenum}
	
	\item
	\item
	\item
	\begin{longenum}
		\item Das \Gls{Zimmer} ist bis zum gewünschten Ablaufdatum nicht frei:
		\begin{longenum}
			\item Das System schlägt ein anderes \Gls{Zimmer} in der selben Kategorie vor, welches ab dem aktuellen Abreisedatum frei ist.
			\item \emph{weiter mit Basisablauf Punkt 6}
		\end{longenum}
		
		\item Es sind keine \Gls{Zimmer} in der selben Kategorie frei bis zum gewünschten Abreisedatum:
		\begin{longenum}
			\item Das System gibt alle freien \Gls{Zimmer}, die für die restliche Verweildauer des \Gls{Gast}es verfügbar sind, zusammen mit einer Warnung aus.
			\item Der \Gls{Rezeptionist} wählt eines dieser Zimmer aus, und verhandelt falls notwendig mit dem \Gls{Gast} neue Konditionen aus.
			\item \emph{weiter mit Basisablauf Punkt 6}
		\end{longenum}
		
		\item Es sind überhaupt keine \Gls{Zimmer} mehr frei:
		\begin{longenum}
			\item Der \Gls{Rezeptionist} teilt dem \Gls{Gast}n mit, dass eine Verlängerung des Aufenthaltes nicht möglich ist.
			\item \emph{UseCase abbrechen}
		\end{longenum}
	\end{longenum}
	
	\item
	\item
\end{longenum}

\paragraph{Besondere Anforderungen}
\begin{itemize}
	\item keine
\end{itemize}

\paragraph{Benutzerfrequenz}
selten

\paragraph{Offene Punkte}

\newpage
