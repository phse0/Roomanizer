\subsubsection{Reservierung stornieren - Vertragspartner}

\paragraph{Kurzbeschreibung}
Der \Gls{Vertragspartner} nimmt Kontakt mit der \Gls{Rezeption} auf. Der \Gls{Vertragspartner} gibt die Daten der zu stornierenden Reservierung an. Der \Gls{Rezeptionist} sucht die entsprechende \Gls{Reservierung}. Er teilt dem \Gls{Vertragspartner} die getroffenen Vereinbarungen mit. Dieser ist damit einverstanden und der \Gls{Rezeptionist} storniert die \Gls{Reservierung}.

\paragraph{Stakeholders und Akteure}
\begin{itemize}
\item \Gls{Rezeptionist} - Einfache und schnelle Stornierung von Reservierungen
\item \Gls{Kunde} - Reibungsloses stornieren von Reservierungen
\end{itemize}

\paragraph{Vorbedingung}
\begin{itemize}
\item Allgemeine Vorbedingungen
\item Es besteht eine \Gls{Reservierung} von diesem \Gls{Vertragspartner}
\end{itemize}

\paragraph{Nachbedingung}
\begin{itemize}
\item Die \Gls{Reservierung} wurde storniert.
\end{itemize}

\paragraph{Basisablauf}
\begin{enumerate}
\item Ein \Gls{Vertragspartner} nimmt Kontakt mit der \Gls{Rezeption} auf und gibt seine \Gls{Reservierungsnummer} an.
\item Der \Gls{Rezeptionist} sucht im System anhand der \Gls{Reservierungsnummer} die \Gls{Reservierung}.
\item Das System zeigt die Reservierungsinformationen an.
\item Der \Gls{Rezeptionist} teilt dem \Gls{Vertragspartner} die bereits angefallenen Kosten laut Vertrag mit.
\item Der \Gls{Vertragspartner} begleicht die angefallenen Kosten.
\item Der \Gls{Rezeptionist} storniert die Reservierung des \Gls{Vertragspartner}s.
\end{enumerate}

\paragraph{Alternativer Ablauf}
\begin{longenum}
	\item
	\begin{longenum}
		\item Der \Gls{Vertragspartner} kennt die \Gls{Reservierungsnummer} nicht:
		\begin{longenum}
			\item Der \Gls{Vertragspartner} gibt stattdessen den Namen der Firma an.
			\item Der \Gls{Rezeptionist} sucht die \Gls{Reservierung} im System anhand des Namens der Firma.
			\begin{longenum}
				\item Das System findet keine \Gls{Reservierung} unter dem angegebenem Namen:
				\begin{longenum}
					\item \emph{UseCase abbrechen}
				\end{longenum}
			\end{longenum}
			\item \emph{weiter mit Basisablauf Punkt 3}
		\end{longenum}
	\end{longenum}
	
	\item
	\item
	\item
	\item
\end{longenum}

\paragraph{Besondere Anforderungen}

\paragraph{Technologie und Daten Variationsliste}

\paragraph{Benutzerfrequenz}
selten

\paragraph{Offene Punkte}

\newpage
