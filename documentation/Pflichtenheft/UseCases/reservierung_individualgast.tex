\subsubsection{Reservierung aufnehmen - Individualgast}
\label{UseCase_ReservierungAufnehmenIndividualgast}

\paragraph{Kurzbeschreibung}
Eine Person tritt in Kontakt mit dem Hotel und will ein \Gls{Zimmer} reservieren. Der \Gls{Mitarbeiter} gibt die Daten der Person im System ein. Die Person gibt dann die gewünschte Kategorie bzw. die \Gls{Zimmernummer} und den Zeitraum an. Es wird geprüft ob in diesem Zeitraum und der Kategorie ein bzw. das \Gls{Zimmer} verfügbar wäre. Dann wird die Person über den Preis, welcher den momentanen Listenpreisen entnommen wird, und die verfügbaren \Gls{Zusatzleistung}en informiert. Danach wird ein \Gls{Optionsdatum} mit der Person vereinbart. Der \Gls{Mitarbeiter} trägt daraufhin die \Gls{Reservierung} im System ein und teilt der Person die \Gls{Reservierungsnummer} mit.

\paragraph{Stakeholders und Akteure}
\begin{itemize}
	\item \Gls{Rezeptionist} - Schnelle und zuverlässige Bearbeitung der
	\Gls{Reservierung}, Konzentration auf die Person, nicht auf den Bildschirm
	\item Person - Reibungsloser, zügiger Ablauf und Erhalten der \Gls{Reservierungsnummer}
\end{itemize}

\paragraph{Vorbedingung}
\begin{itemize}
	\item Allgemeine Vorbedingungen
\end{itemize}

\paragraph{Nachbedingung}
\begin{itemize}
	\item Die Person muss eine Reservierungsnummer erhalten haben
\end{itemize}

\paragraph{Basisablauf}
\begin{enumerate}
	\item Eine Person tritt mit dem Hotel in Kontakt.
	\item Der \Gls{Rezeptionist} gibt die Daten der Person in das System ein.
	\item Die Person gibt die gewünschte Zimmerkategorie und einen bestimmten Zeitraum für ihren Aufenthalt an. 
    \item Der \Gls{Rezeptionist} prüft ob im System ein Zimmer der gwünschten Kategorie für den bestimmten Zeitraum vorhanden ist. 
	\item Der \Gls{Rezeptionist} teilt der Person den Listenpreis mit.
	\item Die Person akzeptiert den Preis, welcher dann ins System eingetragen wird.
	\item Der \Gls{Rezeptionist} trägt die gewünschten \Gls{Zusatzleistung}en der Person ein.
	\item Der \Gls{Rezeptionist} teilt der Person das \Gls{Optionsdatum} und die \Gls{Reservierungsnummer} mit.
\end{enumerate}

\paragraph{Alternativer Ablauf}
\begin{longenum}
	\item
	\item
	\item
	\begin{longenum}
			\item Die Person gibt statt der Zimmerkategorie ihre gewünschte \Gls{Zimmernummer} an:
			\begin{longenum}
				\item Der \Gls{Rezeptionist} überprüft, ob das entsprechende \Gls{Zimmer} für den gewünschten Zeitraum frei ist.
				\begin{longenum}
					\item Das \Gls{Zimmer} ist nicht frei:
					\begin{longenum}
						\item Der \Gls{Rezeptionist} sucht im System ein \Gls{Zimmer} mit der selben Kategorie.
						\item \emph{weiter mit Basisablauf Punkt 5}
					\end{longenum}
				\end{longenum}

				\item \emph{weiter mit Basisablauf Punkt 5}
			\end{longenum}
				
	\end{longenum}
	\item
	\begin{longenum}
		\item In dem gewünschten Zeitraum ist kein \Gls{Zimmer} in der gewünschten Kategorie frei.
		\begin{longenum}
			\item Das System schlägt ein \Gls{Zimmer} in einer anderen Kategorie vor:
			\begin{longenum}
				\item Gar kein \Gls{Zimmer} ist frei:
				\begin{longenum}
					\item \emph{Usecase abbrechen}
				\end{longenum}
			\end{longenum}
			\item \emph{weiter mit Basisablauf Punkt 5}
		\end{longenum}
	\end{longenum}
	\item
	\item
	\item
	\item
\end{longenum}

\paragraph{Besondere Anforderungen}
\begin{itemize}
	\item keine
\end{itemize}

\paragraph{Benutzerfrequenz}
sehr hoch

\newpage
