\subsubsection{Reservierung ändern}

\paragraph{Kurzbeschreibung}
Der \Gls{Rezeptionist} sucht mit dem System nach der \Gls{Reservierung} anhand der Reservierungsdaten. Das System listet die \Gls{Reservierung}(en) auf,  der \Gls{Rezeptionist} wählt die zu ändernde \Gls{Reservierung} aus. Der \Gls{Rezeptionist} kann jetzt über das System die veränderbaren Daten der \Gls{Reservierung} bearbeiten, das System zeigt dabei auch eine \Gls{Belegungsvorschau} an. Bei einer Preisänderung muss das System prüfen ob der \Gls{Rezeptionist} dazu berechtigt ist. Nach der erfolgten Änderung speichert das System diese Änderung der \Gls{Reservierung}.

\paragraph{Stakeholders und Akteure}
\begin{itemize}
	\item \Gls{Rezeptionist} - Schnell die \Gls{Reservierung} aktualisieren
	\item \Gls{Gast} - Änderungen sollen zuverlässig bearbeitet werden.
\end{itemize}

\paragraph{Vorbedingung}
\begin{itemize}
	\item Die \Gls{Reservierung} muss bereits existieren
\end{itemize}

\paragraph{Nachbedingung}
\begin{itemize}
	\item Die Daten der \Gls{Reservierung} sollen wie gewünscht geändert worden sein
\end{itemize}

\paragraph{Basisablauf}
\begin{enumerate}
	\item Der \Gls{Rezeptionist} sucht im System anhand der \Gls{Reservierungsnummer} die \Gls{Reservierung}.
	\item Das System zeigt die gewünschte \Gls{Reservierung} mit einer \Gls{Belegungsvorschau} an.
	\item Der \Gls{Rezeptionist} bearbeitet die Daten der \Gls{Reservierung}.
	\item Der \Gls{Rezeptionist} speichert die Änderungen im System.
\end{enumerate}

\paragraph{Alternativer Ablauf}
\begin{longenum}
	\item
	\begin{longenum}
		\item Der Rezeptionist kennt die \Gls{Reservierungsnummer} nicht:
		\begin{longenum}
			\item Der \Gls{Rezeptionist} sucht im System anhand alternativer Reservierungsdaten nach der \Gls{Reservierung}.
			\item Das System gibt eine gefilterte Menge der \Gls{Reservierung}en zurück.
			\begin{longenum}
				\item Der \Gls{Rezeptionist} kann die gesuchte Reservierung nicht finden:
				\begin{longenum}
					\item \emph{UseCase abbrechen}
				\end{longenum}
			\end{longenum}
			\item Der \Gls{Rezeptionist} wählt die gewünschte Reservierung zur Bearbeitung aus.
			\item \emph{weiter mit Basisablauf Punkt 2}
		\end{longenum}
	\end{longenum}
	\item
	\item
	\begin{longenum}
		\item Der \Gls{Rezeptionist} hat nicht die notwendigen Berechtigungen um die \Gls{Reservierung} zu bearbeiten.
		\begin{longenum}
			\item \emph{UseCase abbrechen}
		\end{longenum}
	\end{longenum}
	\item
\end{longenum}

\paragraph{Besondere Anforderungen}
\begin{itemize}
	\item keine
\end{itemize}

\paragraph{Benutzerfrequenz}
mittel bis selten

\paragraph{Offene Punkte}

\newpage

