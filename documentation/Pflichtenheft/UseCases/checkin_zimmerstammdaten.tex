\subsubsection{Zimmerstammdaten anlegen}

\paragraph{Kurzbeschreibung}
Der Mitarbeiter sucht per Zimmernummer im System und das System bietet ihm an ein neues Zimmer mit der eingegebenen Nummer zu erstellen. Der Mitarbeiter trägt Daten eines Zimmers in das System ein. Um das neue Zimmer zu speichern, bestätigt der Mitarbeiter den Dialog.

\paragraph{Stakeholders und Akteure}
\begin{itemize}
\item \Gls{Backoffice-Mitarbeiter} - Benötigt Zimmerstammdaten für Statistiken usw.
\item \Gls{Rezeptionist} - Braucht Zimmerstammdaten für fast alle seine Interaktionen mit dem System
\item \Glspl{Gast} - Reibungsloser Ablauf und schnelle Bearbeitung seiner Wünsche
\end{itemize}

\paragraph{Vorbedingung}


\paragraph{Nachbedingung}

\paragraph{Basisablauf}
\begin{enumerate}
\item Mitarbeiter öffnet Zimmersuche im System
\item Der Mitarbeiter tippt die Nummer des neuen Zimmers ein
\item Das System zeigt an, dass es noch kein Zimmer mit dieser Nummer gibt
\item Der Mitarbeiter gibt an, dass er ein neues Zimmer mit dieser Nummer anlegen will
\item Der Mitarbeiter gibt alle nötigen Daten für das Anlegen des Zimmers ein
\item Der Mitarbeiter bestätigt seine Eingaben

\paragraph{Alternativer Ablauf}
\begin{longenum}
	\item
	\begin{longenum}
		\item Das System zeigt an, dass es bereits ein Zimmer gibt mit dieser Nummer und der Mitarbeiter will die Stammdaten des Zimmers bearbeiten
		\begin{longenum}
			\item Der Mitarbeiter ändert die entsprechenden Daten des Zimmers 
			\item \emph{weiter mit Basisablauf Punkt 6}
		\end{longenum}
	\end{longenum}
	
	\item
	\item
	\item
	\begin{longenum}
		\item Das System zeigt an, dass es bereits ein \Gls{Zimmer} gibt mit dieser Nummer und der Mitarbeiter will dieses Zimmer löschen
		\begin{longenum}
			\item Der Mitarbeiter wählt aus, dass er das Zimmer löschen will
			\item \emph{weiter mit Basisablauf Punkt 6}
		\end{longenum}
		
		
\end{longenum}

\paragraph{Besondere Anforderungen}

\paragraph{Technologie und Daten Variationsliste}

\paragraph{Benutzerfrequenz}
Sehr selten

\paragraph{Offene Punkte}

\newpage
