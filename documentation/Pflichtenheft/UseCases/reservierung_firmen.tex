\subsubsection{Reservierung aufnehmen - Firmen}

\paragraph{Kurzbeschreibung}  
Die Firma tritt mit dem Hotel in Kontakt und will eine bestimmte \Gls{Zimmer}anzahl reservieren. Es wird geprüft ob die Firma schon in der Gästekartei erfasst worden ist und ob die Preise und Zahlungsmodalitäten schon verhandelt wurden. Wenn dies erledigt wurde, wird geprüft ob für den angegeben Zeitraum und in der angegebenen Menge \Gls{Zimmer} in der gewünschten Kategorie vorhanden sind. Es werden die \Gls{Zusatzleistung} ausgehandelt und ein \Gls{Optionsdatum} bestimmt. Der \Gls{Rezeptionist} trägt daraufhin die \Gls{Reservierung} im System ein.

\paragraph{Stakeholders und Akteure}
\begin{itemize}
	\item \Gls{Rezeptionist} - Schnelle und zuverlässige Bearbeitung der \Gls{Reservierung}, Konzentration auf die Person, nicht auf den Bildschirm
	\item Firma - Reibungsloser, zügiger Ablauf und Reservierungsnummmer erhalten
\end{itemize}

\paragraph{Vorbedingung}
\begin{itemize}
	\item Die Firma muss bereits ein Vertragspartner des Hotels sein
\end{itemize}

\paragraph{Nachbedingung}
\begin{itemize}
	\item Die Firma muss eine \Gls{Reservierungsnummer} erhalten haben
\end{itemize}

\paragraph{Basisablauf}
\begin{enumerate}
	\item Eine Firma tritt mit dem Hotel in Kontakt.
	\item Der \Gls{Rezeptionist} gibt die Daten der Firma in das System ein.
	\item Die Firma gibt eine gewünschte Zimmeranzahl bzw. gewünschten \Gls{Zimmer} und Kategorie und einen bestimmten Zeitraum für ihren \Gls{Aufenthalt} an. 
    \item Der \Gls{Rezeptionist} prüft ob im System die gewünschte Zimmeranzahl für den bestimmten Zeitraum vorhanden ist. 
	\item Der ausgehandelte Preis für die \Gls{Zimmer} wird vom \Gls{Rezeptionist} der Firma mitgeteilt.
	\item Die Firma teilt dem \Gls{Rezeptionist}en die gewünschten Zusatzleistungen mit.
	\item Der \Gls{Rezeptionist} teilt der Firma das Optionsdatum und die Reservierungsnummer mit.
\end{enumerate}

\paragraph{Alternativer Ablauf}
\begin{longenum}
	\item
	\item
	\item
	\item
	\begin{longenum}
		\item Ein gewünschtes \Gls{Zimmer} ist für den gewünschten Zeitraum nicht verfügbar.
		\begin{longenum}
			\item Der \Gls{Rezeptionist} teilt der Firma ein anders \Gls{Zimmer} zu.
			\item \emph{weiter mit Basisablauf Punkt 5}
		\end{longenum}
	\end{longenum}
	\item
	\item
	\item
\end{longenum}

\paragraph{Besondere Anforderungen}
\begin{itemize}
	\item keine
\end{itemize}

\paragraph{Benutzerfrequenz}
sehr hoch

\newpage

