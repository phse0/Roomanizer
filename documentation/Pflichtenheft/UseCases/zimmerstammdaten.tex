\subsubsection{Zimmerstammdaten verwalten}

\paragraph{Kurzbeschreibung}
Der \Gls{Mitarbeiter} sucht per \Gls{Zimmernummer} im System und das System
bietet ihm an, ein neues \Gls{Zimmer} mit der eingegebenen Nummer zu erstellen.
Der \Gls{Mitarbeiter} trägt die Daten eines \Gls{Zimmer}s in das System ein. Um
das neue \Gls{Zimmer} zu speichern, bestätigt der \Gls{Mitarbeiter} den Dialog.

\paragraph{Stakeholders und Akteure}
\begin{itemize}
	\item \Gls{Backoffice}-\Gls{Mitarbeiter} - Benötigt Zimmerstammdaten für Statistiken usw.
	\item \Gls{Rezeptionist} - Braucht Zimmerstammdaten für fast alle seine Interaktionen mit dem System
	\item \Glspl{Gast} - Reibungsloser Ablauf und schnelle Bearbeitung seiner Wünsche
\end{itemize}

\paragraph{Vorbedingung}


\paragraph{Nachbedingung}

\paragraph{Basisablauf}
\begin{enumerate}
	\item \Gls{Mitarbeiter} öffnet Zimmersuche im System.
	\item Der \Gls{Mitarbeiter} tippt die Nummer des neuen \Gls{Zimmer}s ein.
	\item Das System zeigt an, dass es noch kein \Gls{Zimmer} mit dieser Nummer gibt.
	\item Der \Gls{Mitarbeiter} gibt an, dass er ein neues \Gls{Zimmer} mit dieser Nummer anlegen will.
	\item Der \Gls{Mitarbeiter} gibt alle nötigen Daten für das Anlegen des \Gls{Zimmer}s ein.
	\item Der \Gls{Mitarbeiter} bestätigt seine Eingaben.
\end{enumerate}

\paragraph{Alternativer Ablauf}
\begin{longenum}
	\item
	\begin{longenum}
		\item Das System zeigt an, dass es bereits ein \Gls{Zimmer} gibt mit dieser Nummer und der \Gls{Mitarbeiter} will die \Gls{Stammdaten} des \Gls{Zimmer}s bearbeiten.
		\begin{longenum}
			\item Der \Gls{Mitarbeiter} ändert die entsprechenden Daten des \Gls{Zimmer}s. 
			\item \emph{weiter mit Basisablauf Punkt 6}
		\end{longenum}
	\end{longenum}
	
	\item
	\item
	\item
	\begin{longenum}
		\item Das System zeigt an, dass es bereits ein \Gls{Zimmer} gibt mit dieser Nummer und der \Gls{Mitarbeiter} will dieses Zimmer löschen
		\begin{longenum}
			\item Der \Gls{Mitarbeiter} wählt aus, dass er das \Gls{Zimmer} löschen will.
			\item \emph{weiter mit Basisablauf Punkt 6}
		\end{longenum}
	\end{longenum}
\end{longenum}

\paragraph{Besondere Anforderungen}
\begin{itemize}
	\item keine
\end{itemize}

\paragraph{Benutzerfrequenz}
Sehr selten

\paragraph{Offene Punkte}

\newpage
