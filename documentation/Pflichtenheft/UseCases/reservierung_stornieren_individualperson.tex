\subsubsection{Reservierung stornieren - Individualperson}

\paragraph{Kurzbeschreibung}
Ein \Gls{Kunde}, welcher eine \Gls{Reservierung} aufgegeben hat, nimmt Kontakt mit der \Gls{Rezeption} auf. Er muss seine Reservierungsdaten angeben. Daraufhin sucht der \Gls{Rezeptionist} die entsprechende \Gls{Reservierung} des \Gls{Kunde}. Der \Gls{Rezeptionist} teilt dem \Gls{Kunde} die laut gesetzlichen Bestimmungen bereits angefallenen Kosten mit, und die Zahlungsmodalitäten werden laut den Stammdaten erfassten Bedingungen festgelegt. Der \Gls{Kunde} begleicht die angefallenen Kosten und der \Gls{Rezeptionist} storniert die \Gls{Reservierung}.

\paragraph{Stakeholders und Akteure}
\begin{itemize}
\item \Gls{Rezeptionist} - Einfache und schnelle Stornierung von Reservierungen
\item \Gls{Kunde} - Reibungsloses stornieren von Reservierungen
\end{itemize}

\paragraph{Vorbedingung}
\begin{itemize}
\item Allgemeine Vorbedingungen
\item Es besteht eine \Gls{Reservierung} ausgestellt auf den Namen dieser Person
\end{itemize}

\paragraph{Nachbedingung}
\begin{itemize}
\item Die \Gls{Reservierung} wurde storniert.
\end{itemize}

\paragraph{Basisablauf}
\begin{enumerate}
\item Ein \Gls{Kunde} nimmt Kontakt mit der \Gls{Rezeption} auf und gibt seine \Gls{Reservierungsnummer} an.
\item Der \Gls{Rezeptionist} sucht im System anhand der \Gls{Reservierungsnummer} die \Gls{Reservierung} des \Gls{Kunde}.
\item Das System zeigt die Reservierungsinformationen an.
\item Der \Gls{Kunde} will die \Gls{Reservierung} früher als 4 Wochen vor Ankunft stornieren und muss deshalb keine Kosten begleichen.
\item Der \Gls{Rezeptionist} storniert die Reservierung des \Gls{Kunde}n.
\end{enumerate}

\paragraph{Alternativer Ablauf}
\begin{longenum}
	\item
	\begin{longenum}
		\item Der \Gls{Kunde} kennt seine \Gls{Reservierungsnummer} nicht:
		\begin{longenum}
			\item Der \Gls{Kunde} gibt stattdessen seinen Namen an.
			\item Der \Gls{Rezeptionist} sucht die \Gls{Reservierung} im System anhand des Namens.
			\begin{longenum}
				\item Das System findet keine \Gls{Reservierung} mit dem angegebenem Namen:
				\begin{longenum}
					\item \emph{UseCase abbrechen}
				\end{longenum}
			\end{longenum}
			\item \emph{weiter mit Basisablauf Punkt 3}
		\end{longenum}
	\end{longenum}
	
	\item
	\item
	\item
	\begin{longenum}
		\item Der \Gls{Kunde} will die Reservierung zwischen 28 und 15 Tagen vor Ankunft die Reservierung stornieren:
		\begin{longenum}
			\item Der \Gls{Kunde} ist verpflichtet, 30\% vom Betrag zu bezahlen.
			\item \emph{weiter mit Basisablauf Punkt 5}
		\end{longenum}
		\item Der \Gls{Kunde} will die Reservierung innerhalb von 15 Tagen vor Ankunft die Reservierung stornieren:
		\begin{longenum}
			\item Der \Gls{Kunde} muss die erste Nacht komplett bezahlen.
			\item \emph{weiter mit Basisablauf Punkt 5}
		\end{longenum}
	\end{longenum}
	
	\item
\end{longenum}

\paragraph{Besondere Anforderungen}

\paragraph{Technologie und Daten Variationsliste}

\paragraph{Benutzerfrequenz}
öfters

\paragraph{Offene Punkte}

\newpage
