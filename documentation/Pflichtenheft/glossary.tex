\newglossaryentry{Gast}{name=Gast, plural=Gäste, description={Eine Person, die die Leistungen des Hotels in Anspruch nimmt}}
\newglossaryentry{Rezeptionist}{name=Rezeptionist, description={Ein Mitarbeiter des Hotels, welcher Personen im Eingangsbereich empfängt, und deren Anfragen bearbeitet}}
\newglossaryentry{Reservierung}{name=Reservierung, plural=Reservierungen, description={Eine Vormerkung für die Nutzung eines Zimmer für einen gewissen Zeitabschnitt in der Zukunft}}
\newglossaryentry{Zimmer}{name=Zimmer, description={In unserem System ist ein Zimmer ein Raum, welcher als Nächtigungsmöglichkeit für Gäste eingerichtet ist}}
\newglossaryentry{Reservierungsnummer}{name=Reservierungsnummer, description={Eine Nummer zur eindeutigen Identifikation einer Reservierung}}
\newglossaryentry{Checkin}{name=Checkin, description={Der Prozess der abgearbeitet wird, wenn ein Gast an die Rezeption gelangt}}
\newglossaryentry{Rezeption}{name=Rezeption, description={Die Stelle an der Personen empfangen werden, meistens im Eingangsbereich zu finden}}
\newglossaryentry{Zusatzleistung}{name=Zusatzleistung, plural=Zusatzleistungen, description={Eine Leistung die standardmäßig nicht inkludiert ist, und über ein zusätzliches Entgelt in Anspruch genommen werden kann}}
\newglossaryentry{Belegungsnummer}{name=Belegungsnummer, description={Eine Nummer, die zur Identifikation von Rechnungen einzelner Gäste auf dem gleichen Zimmer dient.}}
\newglossaryentry{Reiseleiter}{name=Reiseleiter, description={Die Person, die für eine Reisegruppe verantwortlich ist, und auf dessen Namen die Reservierung lautet}}
\newglossaryentry{Zimmernummer}{name=Zimmernummer, description={Eine eindeutige Identifikation für ein Zimmer in einem Hotel}}
\newglossaryentry{Zwischenrechnung}{name=Zwischenrechnung, description={Eine Aufstellung aller bis zu diesem Zeitpunkt angefallenen Konsumationen}}
\newglossaryentry{Rechnung}{name=Rechnung, plural=Rechnungen, description={Ein Beleg für die Inanspruchnahme von gewissen Leistungen und deren Begleichung im Hotel}}
\newglossaryentry{Checkout}{name=Checkout, description={Der Prozess der abgearbeitet wenn der Gast seinen Aufenthalt im Hotel beendet}}
\newglossaryentry{Optionsdatum}{name=Optionsdatum, description={Datum, bis zu welchem die Reservierung spätestens bestätigt sein muss}}
\newglossaryentry{Belegungsvorschau}{name=Belegungsvorschau, description={Eine Vorschau des Reservierungsstandes pro Kategorie}}
\newglossaryentry{Vertragspartner}{name=Vertragspartner, description={Eine Firma, welche einen Vertrag mit dem Hotel hat und spezielle Konditionen ausgehandelt hat}}
\newglossaryentry{Arrangement}{name=Arrangement, description={Ein Sonderangebot, welches aus dem Aufenthalt plus mehreren Zusatzleistungen gebündelt wird}}
\newglossaryentry{Kunde}{name=Kunde, plural=Kunden, description={Die Person, auf dessen Namen die Reservierung läuft, und welche die Rechnung für den Aufenthalt bezahlt}}
\newglossaryentry{Mitarbeiter}{name=Mitarbeiter, description={Sind Personen, welche in einem Arbeitsverhältnis mit dem Hotel stehen}}
\newglossaryentry{Backoffice}{name=Back-Office, description={Umfasst alle Tätigkeiten, bei denen der Gast nicht direkt involviert ist}}
\newglossaryentry{Frontoffice}{name=Front-Office, description={Umfasst alle Tätigkeiten, die an der Rezeption in Kooperation mit dem Gast geschehen}}
\newglossaryentry{Kontingent}{name=Kontingent, plural=Kontingente, description={Ist eine größere Reservierung für eine gewisse Anzahl von Zimmern von einem Reisebüro}}
\newglossaryentry{Reisebuero}{name=Reisebüro, plural=Reisebüros, description={Ist ein Unternehmen, welches diverse Reisen für ihre Kunden organisiert}}
\newglossaryentry{Report}{name=Report, plural=Reports, description={Eine Zusammenfassung einer großen Anzahl von Daten, die übersichtlich und statistisch aufgearbeitet wurden}}
\newglossaryentry{Stammdaten}{name=Stammdaten, description={Eine spezielle Art von Daten, welche sich eher selten ändern}}
\newglossaryentry{Aufenthalt}{name=Aufenthalt, plural=Aufenthalte, description={Der Zeitraum, in welchem ein Gast in dem Hotel wohnt}}
\newglossaryentry{Saison}{name=Saison, description={Teilt das Jahr in einzelne, preislich unterschiedliche, Zeiträume ein}}
