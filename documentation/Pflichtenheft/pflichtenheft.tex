\documentclass[10pt,a4paper,titlepage]{article}
%packages to use
\usepackage[utf8]{inputenc}
\usepackage[ngerman]{babel}
\usepackage{fancyhdr}
\usepackage{hyperref}
\usepackage[xindy]{glossaries}
\usepackage[T1]{fontenc}

%preamble
\pagestyle{fancy}
\makeglossaries

%glossary entries

%Real Document
\begin{document}

%Footer/Header Definitions
\fancyhf{}

\lhead{\leftmark}
\rhead{Seite \thepage}

%Some Options
\setcounter{secnumdepth}{5}
\setcounter{tocdepth}{2}

%Title page
\title{Pflichtenheft Roomanizer}
\author{Ramon Lopez \and Rafael Neumann \and Daniel Rotter \and Andreas Sinz \and Philipp Steiner}
\date{\today}
\maketitle

%Content
\tableofcontents
\newpage

\section{Einführung}

\newpage

\section{Stakeholder- und Benutzerbeschreibung}

\subsection{Stakeholder\slash{}Benutzer Zusammenfassung}
\begin{tabular}{|l|l|l|}
    \hline
    \textbf{Name} & \textbf{Rolle\slash{}Funktion} & \textbf{Interessiert an} \\
    \hline
    Geschäftsführer &  &  \\
    \hline
    Manager &  & \\
    \hline
    Rezeptionist &  &  \\
    \hline
    Koch &  &  \\
    \hline
    Zimmermädchen &  &  \\
    \hline
    Tourismuskauffrau &  &  \\
    \hline
    Gast &  &  \\
    \hline
    Auftraggeber &  &  \\
    \hline
    Entwickler &  &  \\
    \hline
\end{tabular}

\newpage

\section{Produkt Überblick}

\newpage

\section{Domänenmodell}

\newpage

\section{Dynamisches Modell}

\newpage

\section{Nonfunktionale Anforderungen}

\newpage

\section{Iterationsplan (Timeboxes)}

\newpage

%Glossar
\addcontentsline{toc}{section}{Glossar}
\printglossary[title=Glossar,toctitle=GLOSSAR]
\end{document}
