\documentclass[10pt,a4paper,titlepage]{article}
%packages to use
\usepackage[utf8]{inputenc}
\usepackage[ngerman]{babel}
\usepackage{fancyhdr}
\usepackage{hyperref}
\usepackage[xindy]{glossaries}
\usepackage[T1]{fontenc}
\usepackage{enumitem}
\usepackage{graphicx}

%preamble
\newlist{longenum}{enumerate}{6}
\setlist[longenum,1]{label=\arabic*.}
\setlist[longenum,2]{label=\alph*)}
\setlist[longenum,3]{label=\arabic*.}
\setlist[longenum,4]{label=\alph*)}
\setlist[longenum,5]{label=\arabic*.}
\setlist[longenum,6]{label=\alph*)}
\pagestyle{fancy}
\makeglossaries

%glossary entries
\input glossary.tex

%Real Document
\begin{document}

%Footer/Header Definitions
\fancyhf{}

\lhead{\leftmark}
\rhead{Seite \thepage}

%Some Options
\setcounter{secnumdepth}{5}
\setcounter{tocdepth}{3}

%Title page
\title{Pflichtenheft Roomanizer}
\author{Ramon Lopez \and Rafael Neumann \and Daniel Rotter \and Andreas Sinz \and Philipp Steiner}
\date{\today}
\maketitle

%Content
\tableofcontents
\newpage
\section{Einführung}
\subsection{System}
Das Pflichtenheft beschreibt ein EDV - System für ein Hotel, genannt Roomanizer, welches 
folgende Funktion anbietet:
\begin{itemize}
	\item \Gls{Reservierung}
	\item \Gls{Checkin}
	\item \Gls{Checkout}
	\item Zimmerverwaltung
	\item Kontingentverwaltung für \Gls{Reisebuero}
	\item Verwaltung von \Glspl{Kunde} und \Gls{Vertragspartner}
	\item Protokollierung
	\item Statistiken
\end{itemize}
Das System soll die \Gls{Mitarbeiter} des Hotels bei der Verwaltung unterstützen. Dabei ist vorerst vorgesehen, dass das System wirklich nur von den \Gls{Mitarbeiter}n bedient wird.
\subsection{Zweck}
Das Pflichtenheft beschreibt zuerst die Stakeholder, welche zufrieden zu stellen sind, die angestrebten Eigenschaften der Software, sowie das Domänenmodell, welches die Software darzustellen zu versucht. 

Außerdem werden die UseCases, die im Zusammenspiel der Benutzer mit dem System auftreten, genauer beschrieben, als auch die nicht-funktionalen Anforderungen, die an das System gestellt werden.

In diesem Dokument werden alle notwendigen Anforderung beschrieben. Bei Fertigstellung des Systems wird jede hier definierte Anforderung von der Software bereitgestellt.
\subsection{Umfang}
Das Pfichtenheft umfasst alle Arbeitsprozesse welche vom System unterstützt, und von der Domäne benötigt werden. Die Arbeitsprozesse sind in zwei Gruppen eingeordnet:
\begin{enumerate}
\item \Gls{Frontoffice}, das sind alle Abläufe die direkt an der \Gls{Rezeption} mit dem \Gls{Gast} stattfinden.
\item \Gls{Backoffice}, es umfasst alle Abläufe die nicht direkt mit dem physischen \Gls{Gast} zu tun haben.
\end{enumerate}
Die Prozesse werden schriftlich und mit Hilfe von Diagrammen dargestellt.
\subsection{Referenzen}
\begin{itemize}
	\item Projektbeschreibung Projekt Hotel (Roomanizer) 
	\item Requirements Workshop Protokoll, Freitag 16.03.2012
\end{itemize}
\subsection{Überblick}
Dieses Dokument umfasst folgende Bereiche:
\begin{enumerate}
	\item Stakeholder
        \item Produktüberblick
	\item Domänenmodell
        \item Nichtfunktionale Anforderungen
	\item UseCases
        \item Iterationsplan 
\end{enumerate}

\newpage

\section{Stakeholder- und Benutzerbeschreibung}
In diesem Abschnitt werden die Stakeholders, das sind die Personen oder Gruppen die direkt oder indirekt Interesse am Ergebnis des Projektes haben, genauer beschrieben. Es wird ein Profil der Stakeholder und Benutzer die im Projekt involviert sind, ausgearbeitet. Es werden nicht die spezifischen Anforderungen beschrieben, diese werden detailliert in einem anderen Kapitel behandelt. Die in folgender Tabelle beschriebenen Stakeholders bilden somit eine Basis um festzustellen welche Anforderungen berechtigt sind bzw. benötigt werden. 
\subsection{Stakeholder\slash{}Benutzer Zusammenfassung}
\subsubsection{Geschäftsführer}
\begin{tabular}[t]{|p{5cm}|p{5cm}|}
    \hline
    \textbf{Rolle\slash{}Funktion} & \textbf{Interessiert an} \\
    \hline
Vertritt die Gesellschaft gerichtlich und außergerichtlich, leitet die Geschäfte des Hotels und hat gegenüber Dritten eine unbeschränkte Vertretungsmacht. &
    Verschiedene Möglichkeiten zur Kontrolle über die aktuelle Situation des Hotels, z.B. durch Bereitstellung von \Glspl{Report}.\\
    \hline
\end{tabular}
\subsubsection{Manager}
\begin{tabular}[t]{|p{5cm}|p{5cm}|}
    \hline
    \textbf{Rolle\slash{}Funktion} & \textbf{Interessiert an} \\
    \hline
Ist für den finaziellen Erfolg verantwortlich und koordiniert das Personal. &
	Fehlerfreies System, mit welchem man ganz einfach und schnell \Glspl{Reservierung}, Buchungen etc. verwalten kann. \\
    \hline
\end{tabular}
\subsubsection{\Gls{Rezeptionist}}
\begin{tabular}[t]{|p{5cm}|p{5cm}|}
    \hline
    \textbf{Rolle\slash{}Funktion} & \textbf{Interessiert an} \\
    \hline
Empfängt den \Gls{Gast} an der \Gls{Rezeption} kümmert sich um \Gls{Checkin}, \Gls{Checkout} und um die Probleme, Wünsche, Anregungen der \Glspl{Gast}. &
	Schnelles, zuverlässiges System um Kundenwünsche zu erfüllen. Der Fokus während des Gesprächs mit dem \Gls{Kunde}n sollte aber so wenig wie möglich am Bildschirm liegen. \\
    \hline
\end{tabular}
\subsubsection{Koch}
\begin{tabular}[t]{|p{5cm}|p{5cm}|}
    \hline
    \textbf{Rolle\slash{}Funktion} & \textbf{Interessiert an} \\
    \hline
Personeneinsatzplanung, Speisekartengestalten, Anleitung der \Gls{Mitarbeiter} der Küche und Wareneinkauf &
	System mit zuverlässigen Benachrichtigungen an seine \Gls{Mitarbeiter} für die Einsatzplanung, wünschenswert mit Kalendereinteilung und aktueller Warenstand für die Küche. \\
    \hline
\end{tabular}
\subsubsection{Raumpfleger}
\begin{tabular}[t]{|p{5cm}|p{5cm}|}
    \hline
    \textbf{Rolle\slash{}Funktion} & \textbf{Interessiert an} \\
    \hline
Reinigt die \Gls{Zimmer}, füllt die Minibar auf. &
	Erledigt die zugeteilten Aufgaben möglichst schnell, um die \Glspl{Gast} nicht zu stören. \\
    \hline
\end{tabular}
\subsubsection{Tourismuskauffrau}
\begin{tabular}[t]{|p{5cm}|p{5cm}|}
    \hline
    \textbf{Rolle\slash{}Funktion} & \textbf{Interessiert an} \\
    \hline
Arbeitet an der \Gls{Rezeption}, \Gls{Reservierung} und \Gls{Checkin}, Organisation von Veranstaltungen für die \Glspl{Gast}, Einteilung der \Gls{Zimmer} falls nötig. &
	Leicht mit Tastatur bedienbares System welches leicht zu erlernende Tastenkürzeln beinhaltet. \\
    \hline
\end{tabular}
\subsubsection{\Gls{Gast}}
\begin{tabular}[t]{|p{5cm}|p{5cm}|}
    \hline
    \textbf{Rolle\slash{}Funktion} & \textbf{Interessiert an} \\
    \hline
	Konsumiert die Dienstleistungen des Hotels. Nimmt die Leistungen privat oder geschäftlich in kauf. &
	Möchte einen reibungslosen Ablauf beim \Gls{Checkin} und beim \Gls{Checkout}. \\
    \hline
\end{tabular}
\subsubsection{Auftraggeber}
\begin{tabular}[t]{|p{5cm}|p{5cm}|}
    \hline
    \textbf{Rolle\slash{}Funktion} & \textbf{Interessiert an} \\
    \hline
	Er kennt die Anforderungen im Detail. Und stellt stellt sein Wissen dem Entwickler zur Verfügung. Er entscheidet über die Umsetzung. &
	Erfolgreiche Umsetzung und Integration des Produkts. Und die Kosten dabei im Rahmen halten. \\
    \hline
\end{tabular}
\subsubsection{Entwickler}
\begin{tabular}[t]{|p{5cm}|p{5cm}|}
    \hline
    \textbf{Rolle\slash{}Funktion} & \textbf{Interessiert an} \\
    \hline
Setzt die Angaben des Auftraggebers in ein Softwareprodukt um. &
	Zufriedenheit des Auftraggebers bezüglich dem erstellten Produkt.\\
    \hline
\end{tabular}
\subsection{Benutzerumgebung}
Die Benutzerumgebung ist in \Gls{Frontoffice} und \Gls{Backoffice} unterteilt. Der \Gls{Frontoffice}-Bereich ist der Bereich der direkt mit dem physischen \Gls{Gast} und Hotelmitarbeiter in Verbindung steht. Der \Gls{Backoffice}-Bereich ist der Bereich der nicht direkt mit dem physischen \Gls{Gast} in Verbindung steht. Die Umgebung stellt verschiedene Berechtigungen zur Verfügung. Es ist möglich dem \Gls{Mitarbeiter} bestimmte Rechte zu geben bzw. zu entziehen. Die Umgebung bietet Schnittstellen zur Finanzbuchhaltung, Verwaltung von offenen \Glspl{Rechnung} und zur Food and Beverage Verwaltung.

\newpage

\section{Produkt Überblick}
Dieses Kapitel beschreibt die Fähigkeiten und Eigenschaft der Software genauer. Nach einem kurzen Überblick werden die einzelnen Punkte noch näher durchleuchtet.
\subsection{Zusammenfassung Produktfähigkeiten}
\begin{tabular}{|p{5.7cm}|p{5.7cm}|}
	\hline
	\textbf{Produktfähigkeit} &
	\textbf{Stakeholder Nutzen}
	\\
	\hline
	Zimmerreservierung &
	Höhere Effizienz des Front- und Back-Office
	\\
	\hline
	\Gls{Checkin} &
	Schnell und komfortabel für \Gls{Gast} und \Gls{Frontoffice}
	\\
	\hline
	Automatisierte Rechnungslegung &
	\Glspl{Rechnung} sind vor \Gls{Checkout} schon gestellt, Vorgang wird stark beschleunigt
	\\
	\hline
	Verschiedenste Operationen mit \Glspl{Rechnung} (\Gls{Zwischenrechnung}, Rechnungsteilung, Zahlarten usw.) &
	Flexibler Umgang, kommt \Gls{Kunde}n entgegen und ist trotzdem einfach zu bedienen für \Gls{Frontoffice}
	\\
	\hline
	\Gls{Checkout} &
	Schnell und komfortabel für Gast und \Gls{Frontoffice}
	\\
	\hline
	Kontingentverwaltung für \Gls{Reisebuero} &
	\Gls{Backoffice} kann verschiedenste Vertragskonditionen für \Gls{Reisebuero} erstellen und \Glspl{Kontingent} effektive nutzen
	\\
	\hline
	Vertragspartnerverwaltung &
	\Gls{Backoffice} kann verschiedenste Vertragskonditionen für \Gls{Reisebuero} erstellen und \Glspl{Kontingent} effektiv nutzen
	\\
	\hline
	Berechtigungssystem &
	Teile der Applikation sind aus Sicherheits- und Diskretionsgründen nur Angestellten ab einer bestimmten Berechtigungsstufe verfügbar
	\\
	\hline
	Log-System &
	Alle Vorgänge in der Hotelapplikationen werden unter verschiedensten Gesichtspunkten aufgezeichnet
	\\
	\hline
	Statistiken &
	\Gls{Backoffice} und Management können auf Basis der Log-System aufschlussreiche Statistiken und \Glspl{Report} erstellen die in weiterer Folge den Betrieb optimieren
	\\
	\hline
\end{tabular}

\subsection{Produktfähigkeiten/Eigenschaften}
\subsubsection{\Gls{Checkin}}
Roomanizer ist in der Lage \Glspl{Checkin} schnell und zuverlässig zu verarbeiten. Der Akteur der das System bedient kann mittels  verschiedensten Suchbegriffen nach vorhandenen \Glspl{Reservierung} suchen und den \Glspl{Checkin} durchführen. Die \Gls{Reservierung} kann "`On-The-Fly"' noch geändert werden (\Glspl{Zusatzleistung} hinzufügen oder entfernen, Preis ändern, \Gls{Zimmer} ändern, usw.). Weiters kann der Akteur noch die Identität des \Gls{Gast}es mit dem System abgleichen.

\subsubsection{\Gls{Checkout}}
Roomanizer gestaltet den \Gls{Checkout} von einzelnen \Glspl{Gast} oder Gruppen sehr einfach. Weiters können \Gls{Checkout} durch automatische, im Vorfeld, gelegte (Zwischen-)\Glspl{Rechnung} beschleunigt werden.

\subsubsection{\Glspl{Reservierung}}
\Glspl{Reservierung} können mit einem \Gls{Optionsdatum} versehen werden. Weiters bleibt es dem Betrieb offen welche Daten mindestens benötigt werden um eine \Gls{Reservierung} zu machen oder zu bestätigen.

\subsubsection{\Gls{Stammdaten}}
Der \Gls{Mitarbeiter} (mit entsprechender Berechtigungsstufe) wählt eine Kategorie der \Gls{Stammdaten} aus die er bearbeiten will oder in die er neue \Gls{Stammdaten} eintragen will. Eine schnelle Datenbankanbindung macht das editieren der \Gls{Stammdaten} sehr komfortabel.

\subsubsection{\Gls{Aufenthalt} verlängern}
Über die selbe Suche die der Akteur bereits aus dem \Gls{Checkin} kennt, können \Glspl{Aufenthalt} verändert werden. Auch für den Fall, dass das aktuell belegte \Gls{Zimmer} nicht weiter verfügbar ist kann man mit Roomanizer eine passende Lösung finden. 

\subsubsection{Zimmerwechsel}
Der Akteur wählt den aktuellen \Gls{Aufenthalt} des betreffenden \Gls{Gast}es aus und kann diesen in ein passendes \Gls{Zimmer} umbuchen. Auch eventuell nötige Kategoriewechsel, die erst mit dem \Gls{Gast} abgesprochen werden müssen, sind möglich.

\subsubsection{Tages-, Wochen- und Monatsabschluss}
Vor allem für das \Gls{Backoffice} interessant ist die Funktion automatisierte Abschlüsse erstellen zu lassen und diese über die Schnittstelle zur Buchhaltung weiterzuleiten.

\subsubsection{\Glspl{Rechnung}}
Einerseits können jederzeit \Glspl{Zwischenrechnung} erstellt, andererseits können Abschlussrechnungen beim \Gls{Checkout} auch problemlos aufgeteilt werden, um zum Beispiel Geschäftsreisenden entgegenzukommen. Daneben bietet Roomanizer die Möglichkeit an einem festgelegten Zeitpunkt automatische (Zwischen-)\Glspl{Rechnung} zu erstellen. 

\subsubsection{Statistiken}
Es können jederzeit Statistiken und \Glspl{Report} zu Zimmerauslastung, \Glspl{Zusatzleistung}, Gästedaten, Zufriedenheitsfaktoren etc. erstellt und ausgedruckt werden.

\subsubsection{Schnittstelle zur Buchhaltung}
Roomanizer bietet eine Schnittstelle zu verschiedensten Typen von Buchhaltungssoftware indem Rechnungen automatisiert als .csv-Datei weitergeleitet werden.

\subsection{Berechtigungssystem}
Um das System für ein möglichst breites Anwenderspektrum zugänglich zu machen aber trotzdem gewissen Sicherheitsaspekte nicht zu vernachlässigen bietet Roomanizer ein durchgängiges Berechtigungssystem. Der Benutzer loggt sich mit seinen Benutzerdaten ein und erhält dann Zugriff auf die für ihn definierten Funktionen.

\newpage

\section{Domänenmodell}
Hier wird versucht das System des Hotels detailliert darzustellen. Wichtig dabei ist, dass es sich nicht um einen Software-Entwurf, sondern um ein Echtwelt-Abbild handelt.
\subsection{Überblick}
\begin{figure}[h]
	\includegraphics[width=\linewidth]{Images/Domaenenmodell_Uebersicht.png}
	\caption{Übersicht des Domänenmodells}
\end{figure}
\newpage
\subsection{Detailliertes Modell}
\begin{figure}[h]
	\includegraphics[width=\linewidth]{Images/Domaenenmodell.png}
	\caption{Detailliertes Domänenmodell}
\end{figure}
\subsubsection{\Gls{Gast}}
Beim \Gls{Gast} meint man eine Person welche im Hotel Leistungen bezieht. In ihr ist unter anderem der Name und die Anschrift der Person festgehalten.
\subsubsection{\Gls{Reservierung}}
Eine \Gls{Reservierung} besteht aus verschiedenen Reservierungspositionen, welche eine Zimmerkategorie und eine Menge enthalten. Dazu wird noch der \Gls{Kunde} gespeichert, welcher die \Gls{Reservierung} getätigt hat.
\subsubsection{Reservierungsposition}
Die Reservierungsposition gibt an in welcher Kategorie eine bestimmte Anzahl von
\Gls{Zimmer}n reserviert wurde.
\subsubsection{\Gls{Vertragspartner}}
Beim \Gls{Vertragspartner} handelt es sich um eine spezielle Form eines \Gls{Kunde}n.
Er hat einen Vertrag mit dem Hotel abgeschlossen und erhält spezielle Konditionen. Die \Gls{Vertragspartner} werden unterteilt in Firmen und \Glspl{Reisebuero}.
\subsubsection{Firma}
Eine Firma ist ein spezieller \Gls{Vertragspartner}, sie erhält spezielle Konditionen
durch den Vertrag den sie abgeschlossen hat.
\subsubsection{\Gls{Reisebuero}}
Ein \Gls{Reisebuero} ist ein spezieller \Gls{Vertragspartner}, durch den Vertrag der
abgeschlossen wurde kann ein \Gls{Reisebuero} ein bestimmtes \Gls{Kontingent}
zugewiesen bekommen.
\subsubsection{\Gls{Kontingent}}
Bei dem \Gls{Kontingent} handelt es sich um eine spezielle Art von \Glspl{Reservierung} die nur für \Glspl{Reisebuero} verwendet werden. Für sie gelten spezielle Regelungen und werden deshalb getrennt verwaltet.
\subsubsection{Kontingentposition}
Eine Kontingentposition ist eine Aufzählung von \Gls{Zimmer}n einer bestimmten
Kategorie. Sie ist einem \Gls{Kontingent} zugewiesen.
\subsubsection{\Gls{Zimmer}}
Das Hotelzimmer ist ein Raum den ein Kunde gegen Bezahlung in Anspruch nehmen kann.
\subsubsection{Zimmerstatus}
Der Zimmerstatus beschreibt in welcher Lage sich das \Gls{Zimmer} im aktuellen Moment befindet.
\subsubsection{Zimmerkategorie}
Die \Gls{Zimmer} werden in bestimmte Kategorien unterteilt. Von jeder Kategorie ist
nur eine bestimmte Anzahl von \Gls{Zimmer}n vorhanden, welche auch einen Preis
zugewiesen haben.
\subsubsection{\Gls{Kunde}}
Der \Gls{Kunde} ist die Instanz die in direkten Zusammenhang mit einer Rechnung steht.
Einem \Gls{Kunde}n ist eindeutig eine \Gls{Rechnung} zuzuordnen.
\subsubsection{\Gls{Rechnung}}
Eine \Gls{Rechnung} wird erstellt und durch ein Zahlungsmittel beglichen, dann gilt die \Gls{Rechnung} als abgeschlossen. Eine \Gls{Rechnung} hat folgende Bestandteile.
\subsubsection{Rechnungsposition}
Die Rechnungsposition enthält die Anzahl einer bestimmten \Gls{Zusatzleistung} oder
eines Aufenthaltes. Sie wird in einer \Gls{Rechnung} aufgelistet.
\subsubsection{Preis}
Der Preis gibt an wie viel ein \Gls{Zimmer} in einer bestimmten Kategorie kostet. Er
ist unter anderem abhängig von der \Gls{Saison}.
\subsubsection{\Gls{Saison}}
Da es verschiedene \Glspl{Saison} im Jahr gibt, wird mit dieser \Gls{Saison} angegeben
welcher Einfluss auf den Preis etc genommen wird. 
\subsubsection{\Gls{Aufenthalt}}
Der \Gls{Aufenthalt} gibt an welchee \Glspl{Gast} welche \Gls{Zimmer} haben.
Außerdem wird dem \Gls{Aufenthalt} die \Glspl{Zusatzleistung} zugewiesen. Der \Gls{Aufenthalt} ist auch eine eigene Rechnungsposition.
\subsubsection{\Gls{Zusatzleistung}}
Bei einer \Gls{Zusatzleistung} handelt es sich um eine bestimmte Leistung welche nicht
beim normalen \Gls{Aufenthalt} inkludiert ist. Sie wird zusätzlich verrechnet.

\newpage

\section{Dynamisches Modell}
\subsection{Detaillierte Benutzungsfallbeschreibungen}
\input UseCases/checkin_walkin
\input UseCases/checkin_reisegruppe
\input UseCases/checkin_reservierung
\input UseCases/checkout
\input UseCases/reservierung_firmen
\input UseCases/reservierung_individualgast
\input UseCases/reservierung_reisebuero
\input UseCases/reservierung_aendern
\input UseCases/reservierung_bestaetigen
\input UseCases/reservierung_stornieren_individualperson
\input UseCases/reservierung_stornieren_vertragspartner
\input UseCases/aufenthalt_verlaengern
\input UseCases/zimmer_wechsel
\input UseCases/zimmerstammdaten
\input UseCases/gaststammdaten

\newpage

\section{Nichtfunktionale Anforderungen}
Dieser Abschnitt beschreibt die nichtfunktionalen Anforderugen die an die Software gestellt werden. Das heißt wir werden hier nochmals genauer auf diese Anforderungen eingehen, und beschreiben auf welche wir besonderen Fokus legen.
\subsection{Usability}
Natürlich wird versucht die Software selbsterklärend zu gestalten, trotzdem inkludiert das Projekt eine fortlaufende Dokumentation, sowie eine Hilfe, die die Einarbeitungszeit wesentlich verringern soll.

Weiters soll die Software über eine flache Lernkurve verfügen, und somit wenig Schulungsaufwand erzeugen. Das soll durch das Verwenden von bereits bekannten mentalen Modellen und Design Patterns, welche auch in anderer gängiger Software zum Einsatz kommen, erreicht werden.
\subsection{Zuverlässigkeit}
Die Datenbank, auf der das System arbeitet, sollte sich immer einen konsistenten Zustand merken, damit im Fehlerfall ein einfacher Neustart des Systems zur Behebung ausreicht. Trotzdem sollten am besten noch automatisiert Backups erstellt werden, um die Wahrscheinlichkeit eines zu großen Datenverlusts zu begrenzen.
\subsection{Performanz}
Es ist für die Zufriedenheit des \Gls{Kunde}n, und somit auch für den \Gls{Rezeptionist}, äußerst wichtig dass oft vorkommende Tätigkeiten schnell abgehandelt werden können. Dazu zählen vor allem der \Gls{Checkin}, und der \Gls{Checkout}, welche, insofern keine Daten nachzutragen bzw. andere Zusatzaufwände zu erledigen sind, nicht länger als 2 Minuten dauern sollten.

Außerdem ist die ständige Verfügbarkeit des Systems von großer Bedeutung, damit Anfragen von \Gls{Kunde}n bzw. \Glspl{Gast}n sofort bearbeitet werden können. Sollte die Verfügbarkeit für einen gewissen Zeitraum nicht gegeben sein, z.B. wegen Wartungsarbeiten, dann muss dieser Ausfall bereits im Vorherein kommuniziert werden, sodass das Personal sich auf diese unangenehme Situation vorbereiten kann.
\subsection{Unterstützbarkeit}
Um eine möglichst breite Benutzerschaft anzusprechen, wird die Software mehrsprachig gehalten. Leider können wir aus Zeitgründen nicht auf Seh- oder andere Behinderungen eingehen, so dass eingeschränkte Personen auf andere Hilfsmittel wie Screenreader oder ähnliches zurückgreifen müssen.

Das Verwenden einer Maus soll aber schon optional sein, da die Einstiegshürde andernfalls für Menschen mit Sehbehinderung außerordentlich groß wäre. Das heißt dass die Software auch alleine mit der Tastatur zu bedienen sein sollte.
\subsection{Schnittstellen}
\subsubsection{Benutzerschnittstellen}
Die Software wird vom Benutzer über eine grafische Benutzeroberfläche auf handelsüblichen Desktoprechnern bedient.

Das Hauptfenster soll die Grundfunktionalitäten (Reservierungen, Gäste, ...) über einzelne Reiter zur Verfügung stellen. Die Auflistung dieser Elemente erfolgt über Listen, wobei das aktuell ausgewählte Element in einer Vorschau auch editiert werden kann. Über diverse Buttons können verschiedene Aktionen gestartet werden.
\begin{figure}[h]
	\includegraphics[width=\linewidth]{Images/GUI_Overview.png}
	\caption{Übersicht Reservierungen}
\end{figure}

Als kurzes Beispiel sei hier noch ein der Umgang mit \Glspl{Reservierung} erwähnt. Wie bereits erwähnt enthält ein Tab eine Liste mit allen \Glspl{Reservierung}. Bei Auswahl einer \Gls{Reservierung} sieht man rechts eine editierbare Vorschau, und einige Buttons, mit denen man Aktionen auf Bezug des aktuell ausgewählten Element auswählen kann.

\begin{figure}[h]
	\includegraphics[width=\linewidth]{Images/GUI_Dialog.png}
	\caption{Checkin bestätigen}
\end{figure}

Drückt man z.B. den  \Gls{Checkin}-Button, so erhält man ein Dialogfenster, in dem man eine Auflistung aller zugewiesenen \Gls{Zimmer}, und deren Preise hat. All diese Informationen können noch geändert werden, bevor man mit einem Klick die ganzen \Gls{Zimmer} vergibt.
\subsubsection{Software Schnittstellen}
Es sind Schnittstellen zu den bestehenden Systemen (Finanzbuchhaltung, Debitorenbuchhaltung, Food and Beverage Verwaltung) vorzusehen.
\subsubsection{Kommunikationsschnittstellen}
Es ist nicht vorgesehen, dass das System von jedem \Gls{Mitarbeiter} des Hotels zu bedienen ist, da es gerade bei \Gls{Mitarbeiter} in Positionen mit hoher Fluktuation zu teuer wäre, jeden einzelnen dem Umgang mit der Software zu erläutern. Daher werden immer noch diverse Formulare existieren, die von gewissen Berufsgruppen (z.B. Reinigungspersonal) ausgefüllt, und danach von \Gls{Mitarbeiter}n im \Gls{Backoffice} manuell ins System übertragen werden.

\newpage

\section{Iterationsplan (Timeboxes)}
\input iterationsplan

\newpage

%Glossar
\addcontentsline{toc}{section}{Glossar}
\printglossary[title=Glossar,toctitle=GLOSSAR]

\newpage

\listoffigures
\end{document}
