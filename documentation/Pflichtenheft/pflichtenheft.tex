\documentclass[10pt,a4paper,titlepage]{article}
%packages to use
\usepackage[utf8]{inputenc}
\usepackage[ngerman]{babel}
\usepackage{fancyhdr}
\usepackage{hyperref}
\usepackage[xindy]{glossaries}
\usepackage[T1]{fontenc}
\usepackage{enumitem}

%preamble
\newlist{longenum}{enumerate}{6}
\setlist[longenum,1]{label=\arabic*.}
\setlist[longenum,2]{label=\alph*)}
\setlist[longenum,3]{label=\arabic*.}
\setlist[longenum,4]{label=\alph*)}
\setlist[longenum,5]{label=\arabic*.}
\setlist[longenum,6]{label=\alph*)}
\pagestyle{fancy}
\makeglossaries

%glossary entries
\newglossaryentry{Gast}{name=Gast, plural=Gäste, description={Der Gast ist die Person, die die Leistungen des Hotels in Anspruch nimmt}}
\newglossaryentry{Rezeptionist}{name=Rezeptionist, description={Ein Mitarbeiter des Hotels, welcher Personen im Eingangsbereich empfängt, und deren Anfragen bearbeitet}}
\newglossaryentry{Reservierung}{name=Reservierung, description={Eine Reservierung ist eine Vormerkung für die Nutzung eines Zimmer für einen gewissen Zeitabschnitt in der Zukunft}}
\newglossaryentry{Zimmer}{name=Zimmer, description={In unserem System ist ein Zimmer ein Raum, welcher als Nächtigungsmöglichkeit für Gäste eingerichtet ist}}
\newglossaryentry{Reservierungsnummer}{name=Reservierungsnummer, description={Eine Reservierungsnummer ist eine eindeutige Identifikation einer Reservierung}}
\newglossaryentry{Checkin}{name=Checkin, description={Der Prozess der abgearbeitet wird, wenn ein Gast an die Rezeption gelangt}}
\newglossaryentry{Rezeption}{name=Rezeption, description={Die Rezeption ist die Stelle an der Personen empfangen werden, und ist meistens im Eingangsbereich zu finden}}
\newglossaryentry{Zusatzleistung}{name=Zusatzleistung, description={Eine Zusatzleistung ist eine Leistung die standardmäßig nicht inkludiert ist, und über ein zusätzliches Entgelt in Anspruch genommen werden kann}}
\newglossaryentry{Belegungsnummer}{name=Belegungsnummer, description={Die Belegungsnummer dient zur Identifikation von Rechnungen einzelner Gäste auf dem gleichen Zimmer}}
\newglossaryentry{Reiseleiter}{name=Reiseleiter, description={Die Person, die für eine Reisegruppe verantwortlich ist, und auf dessen Namen die Reservierung lautet}}

%Real Document
\begin{document}

%Footer/Header Definitions
\fancyhf{}

\lhead{\leftmark}
\rhead{Seite \thepage}

%Some Options
\setcounter{secnumdepth}{5}
\setcounter{tocdepth}{3}

%Title page
\title{Pflichtenheft Roomanizer}
\author{Ramon Lopez \and Rafael Neumann \and Daniel Rotter \and Andreas Sinz \and Philipp Steiner}
\date{\today}
\maketitle

%Content
\tableofcontents
\newpage

\section{Einführung}

\newpage

\section{Stakeholder- und Benutzerbeschreibung}

\subsection{Stakeholder\slash{}Benutzer Zusammenfassung}
\begin{tabular}{|l|l|l|}
    \hline
    \textbf{Name} & \textbf{Rolle\slash{}Funktion} & \textbf{Interessiert an} \\
    \hline
    Geschäftsführer &  &  \\
    \hline
    Manager &  & \\
    \hline
    Rezeptionist &  &  \\
    \hline
    Koch &  &  \\
    \hline
    Zimmermädchen &  &  \\
    \hline
    Tourismuskauffrau &  &  \\
    \hline
    Gast &  &  \\
    \hline
    Auftraggeber &  &  \\
    \hline
    Entwickler &  &  \\
    \hline
\end{tabular}

\newpage

\section{Produkt Überblick}

\newpage

\section{Domänenmodell}

\newpage

\section{Dynamisches Modell}
\subsection{Detaillierte Benutzungsfallbeschreibungen}
\input UseCases/checkin_reisegruppe
\input UseCases/checkin_reservierung
\input UseCases/zimmer_wechsel

\newpage

\section{Nonfunktionale Anforderungen}

\newpage

\section{Iterationsplan (Timeboxes)}

\newpage

%Glossar
\addcontentsline{toc}{section}{Glossar}
\printglossary[title=Glossar,toctitle=GLOSSAR]
\end{document}
