\documentclass[10pt,a4paper,titlepage]{article}
%packages to use
\usepackage[utf8]{inputenc}
\usepackage[ngerman]{babel}
\usepackage{fancyhdr}
\usepackage{hyperref}
\usepackage[xindy]{glossaries}
\usepackage[T1]{fontenc}
\usepackage{enumitem}

%preamble
\newlist{longenum}{enumerate}{6}
\setlist[longenum,1]{label=\arabic*.}
\setlist[longenum,2]{label=\alph*)}
\setlist[longenum,3]{label=\arabic*.}
\setlist[longenum,4]{label=\alph*)}
\setlist[longenum,5]{label=\arabic*.}
\setlist[longenum,6]{label=\alph*)}
\pagestyle{fancy}
\makeglossaries

%glossary entries
\newglossaryentry{Gast}{name=Gast, plural=Gäste, description={Eine Person, die die Leistungen des Hotels in Anspruch nimmt}}
\newglossaryentry{Rezeptionist}{name=Rezeptionist, description={Ein Mitarbeiter des Hotels, welcher Personen im Eingangsbereich empfängt, und deren Anfragen bearbeitet}}
\newglossaryentry{Reservierung}{name=Reservierung, description={Eine Vormerkung für die Nutzung eines Zimmer für einen gewissen Zeitabschnitt in der Zukunft}}
\newglossaryentry{Zimmer}{name=Zimmer, description={In unserem System ist ein Zimmer ein Raum, welcher als Nächtigungsmöglichkeit für Gäste eingerichtet ist}}
\newglossaryentry{Reservierungsnummer}{name=Reservierungsnummer, description={Eine Nummer zur eindeutigen Identifikation einer Reservierung}}
\newglossaryentry{Checkin}{name=Checkin, description={Der Prozess der abgearbeitet wird, wenn ein Gast an die Rezeption gelangt}}
\newglossaryentry{Rezeption}{name=Rezeption, description={Die Stelle an der Personen empfangen werden, meistens im Eingangsbereich zu finden}}
\newglossaryentry{Zusatzleistung}{name=Zusatzleistung, description={Eine Leistung die standardmäßig nicht inkludiert ist, und über ein zusätzliches Entgelt in Anspruch genommen werden kann}}
\newglossaryentry{Belegungsnummer}{name=Belegungsnummer, description={Eine Nummer, die zur Identifikation von Rechnungen einzelner Gäste auf dem gleichen Zimmer dient.}}
\newglossaryentry{Reiseleiter}{name=Reiseleiter, description={Die Person, die für eine Reisegruppe verantwortlich ist, und auf dessen Namen die Reservierung lautet}}
\newglossaryentry{Zimmernummer}{name=Zimmernummer, description={Eine eindeutige Identifikation für ein Zimmer in einem Hotel}}
\newglossaryentry{Zwischenrechnung}{name=Zwischenrechnung, description={Eine Aufstellung aller bis zu diesem Zeitpunkt angefallenen Konsumationen}}
\newglossaryentry{Rechnung}{name=Rechnung, description={Ein Beleg für die Inanspruchnahme von gewissen Leistungen und deren Begleichung im Hotel}}
\newglossaryentry{Checkout}{name=Checkout, description={Der Prozess der abgearbeitet wenn der Gast seinen Aufenthalt im Hotel beendet}}
\newglossaryentry{Optionsdatum}{name=Optionsdatum, description={Datum, bis zu welchem die Reservierung spätestens bestätigt sein muss}}
\newglossaryentry{Belegungsvorschau}{name=Belegungsvorschau, description={Eine Vorschau des Reservierungsstandes pro Kategorie}}
\newglossaryentry{Vertragspartner}{name=Vertragspartner, description={Eine Firma, welche einen Vertrag mit dem Hotel hat und spezielle Konditionen ausgehandelt hat}}

%Real Document
\begin{document}

%Footer/Header Definitions
\fancyhf{}

\lhead{\leftmark}
\rhead{Seite \thepage}

%Some Options
\setcounter{secnumdepth}{5}
\setcounter{tocdepth}{3}

%Title page
\title{Pflichtenheft Roomanizer}
\author{Ramon Lopez \and Rafael Neumann \and Daniel Rotter \and Andreas Sinz \and Philipp Steiner}
\date{\today}
\maketitle

%Content
\tableofcontents
\newpage

\section{Einführung}
\subsection{System}
Das Pflichtenheft beschreibt ein EDV - System für ein Hotel(Roomanizer), welches 
folgende Funktion anbietet:
\begin{itemize}
	\item Reservierung
	\item Check - In
	\item Check - Out
	\item Zimmerverwaltung
	\item Kontingentverwaltung für Reisebüro
	\item Verwaltung von Kunden und Vertragspartner
	\item Protokollierung
	\item Statistiken
\end{itemize}
\subsection{Zweck}
 Das Pflichtenheft beschreibt alle Anforderung welche im Roomanizer enthalten sein müssen. Bei fertigstellung des Systems wird jede hier definierte Anforderung von der Software bereitgestellt.
\subsection{Umfang}
Das Pfichtenheft umfasst alle Arbeitsprozesse welche vom System unterstützt werden. 
Die Arbeitsprozesse sind in zwei Gruppen eingeordnet:
\begin{enumerate}
\item Front-Office, das sind alle Abläufe die direkt an der Rezeption mit dem Gast stattfinden.
\item Back-Office, es umfasst alle Abläufe die nicht direkt mit dem physischen Gast zu tun haben.
\end{enumerate}
Die Prozesse werden schriftlich und mit Hilfe von Diagrammen dargestellt.
\subsection{Referenzen}
\begin{itemize}
	\item Projektbeschreibung Projekt Hotel (Roomanizer) 
	\item Requirements Workshop Protokoll, Freitag 16.03.2012
\end{itemize}

\subsection{Überblick}
Dieses Document umfasst folgende Bereiche:
\begin{enumerate}
	\item Stakeholder
	\item Domänenmodell
	\item UseCases 
\end{enumerate}

\newpage

\section{Stakeholder- und Benutzerbeschreibung}
Stakeholders sind die Personen oder Gruppen die direkt oder indirekt Interesse am Ergebnis des Projektes haben. In diesem Abschnitt wird ein Profil, der Stakeholder und Benutzer die im Projekt involviert sind, beschrieben. Es werden nicht die spezifischen Anforderungen beschrieben, diese werden detailliert in einem anderen Kapitel behandelt. Die in folgender Tabelle beschriebenen Stakeholders bilden somit eine Basis um festzustellen welche Anforderungen berechtigt sind bzw. benötigt werden. 
\subsection{Stakeholder\slash{}Benutzer Zusammenfassung}
\begin{tabular}{|l|p{4cm}|p{4cm}|}
    \hline
    \textbf{Name} & \textbf{Rolle\slash{}Funktion} & \textbf{Interessiert an} \\
    \hline
    Geschäftsfuehrer & 
Vertritt die Gesellschaft gerichtlich und ausergerichtlich, leitet die Geschäfte des Hotels und hat gegenueber Dritten eine unbeschränkte Vertretungsmacht, auch repräsentative Tätigkeiten &
    Verschiedene Möglichkeiten zur Kontrolle über die aktuelle Situation des Hotels, z.B. durch Bereitstellung von Reports.\\
    \hline
    Manager &
	Ist für den finaziellen Erfolg verantwortlich und koordiniert das Personal &
	Fehlerfreies System, mit welchem man ganz einfach und schnell Reservierungen, Buchungen usw. verwalten kann. \\
    \hline
    Rezeptionist &
	Empfängt den Gast an der Rezeption kümmert sich um Checkin, Check-out und um die Probleme, Wünsche, Anregungen der Gäste. &
	Schnelles, zuverlässiges System um Kundenwünsche zu erfüllen. Der Fokus während des Gesprächs mit dem Kunden sollte aber so wenig wie möglich am Bildschirm liegen. \\
    \hline
    Koch &
Personeneinsatzplanung, Speisekartengestalten, Anleitung der Mitarbeiter der Küche und Wareneinkauf &
	System mit zuverlässigen Benachrichtigungen an seine Mitarbeiter für die Einsatzplanung, wünschenswert mit Kalendereinteilung und aktueller Warenstand für die Küche. \\
    \hline
    Zimmermädchen & 
	Reinigt die Zimmer, füllt die Minibar auf. &
	Sie erledigt Ihre Aufgabe möglichst schnell, um die Gäste nicht zu stören. \\
    \hline
    Tourismuskauffrau &
	Arbeitet an der Rezeption, Reservierung und Check-In, Organisation von Veranstaltungen für die Gaeste, Einteilung der Zimmer falls nötig &
	Leicht mit Tastatur bedienbares System welches leicht zu Erlernende Tastenkürzeln  \\
    \hline
    Gast &
	Konsumiert die Dienstleistungen des Hotels. Ist geschäftlich Unterwegs &
	Im Hotel übernachten um am nächsten Tag an einem Kongress teilzunehmen. \\
    \hline
    Auftraggeber &
	Er kennt die Anforderungen im Detail. Und stellt stellt sein Wissen dem Entwickler zur Verfügung. Er entscheidet über die Umsetzung. &
	Erfolgreiche Umsetzung und Integration des Produkts. Und die Kosten dabei im Rahmen halten. \\
    \hline
    Entwickler &
Setzt die Angaben des Auftraggebers in ein Softwareprodukt um.&
Zufriedenheit des Auftraggebers bezüglich dem erstellten Produkt.\\
    \hline
\end{tabular}

\newpage

\section{Produkt Überblick}
\subsection{Zusammenfassung Produktfähigkeiten}
\begin{tabular}{|p{5.7cm}|p{5.7cm}|}
	\hline
	\textbf{Produktfähigkeit} &
	\textbf{Stakeholder Nutzen}
	\\
	\hline
	Zimmerreservierung &
	Höhere Effizienz des Front- und Backoffice
	\\
	\hline
	Check-In &
	Schnell und komfortabel für Gast und Frontoffice
	\\
	\hline
	Automatisierte Rechnungslegung &
	Rechnungen sind vor Checkout schon gestellt, Vorgagng wird stark beschleunigt
	\\
	\hline
	Verschiedenste Operationen mit Rechnungen (Zwischenrechnung, Rechnungsteilung, Zahlarten usw.) &
	Flexibler Umgang, kommt Kunden entgegen und ist trotzdem einfach zu bedienen für Frontoffice
	\\
	\hline
	Checkout &
	Schnell und komfortabel für Gast und Frontoffice
	\\
	\hline
	Kontingentverwaltung für Reisebüros &
	Backoffice kann verschiedenste Vertragskonditionen für Reisebüros erstellen und Kontingente effektive nutzen
	\\
	\hline
	Vertragspartnerverwaltung &
	Backoffice kann verschiedenste Vertragskonditionen für Reisebüros erstellen und Kontingente effektiv nutzen
	\\
	\hline
	Berechtigungssystem &
	Teile der Applikation sind aus Sicherheits- und Diskretionsgründen nur Angestellten ab einer bestimmten Berechtigungsstufe verfügbar
	\\
	\hline
	Log-System &
	Alle Vorgänge in der Hotelapplikationen werden unter verschiedensten Gesichtspunkten aufgezeichnet
	\\
	\hline
	Statistiken &
	Backoffice und Management können auf Basis der Log-System aufschlussreiche Statistiken und Report erstellen die in weiterer Folge den Betrieb optimieren
	\\
	\hline
\end{tabular}

\newpage

\section{Domänenmodell}
\subsection{Überblick}
\subsection{Detailliertes Modell}
\subsubsection{Gast}
Beim Gast meint man eine Person welche im Hotel Leistungen bezieht. In ihr ist unter anderem der Name und die Anschrift der Person festgehalten.
\subsubsection{Reservierung}
Eine Reservierung besteht aus verschiedenen Reservierungspositionen, welche eine Zimmerkategorie und eine Menge enthalten. Dazu wird noch der Kunde gespeichert, welcher die Resevierung getätigt hat.
\subsubsection{Vertragspartner}
Beim Vertragspartner handelt es sich um eine spezielle Form eines Kunden. Er hat einen Vertrag mit dem Hotel abgeschlossen und erhält spezielle Konditionen. Die Vertragspartner werden unterteilt in Firmen und Reisebüros.
\subsubsection{Kontingent}
Bei dem Kontingent handelt es sich um eine spezielle Art von Reservierung die nur für Reisebüros verwendet werden. Für sie gelten spezielle Regelungen und werden deshalb getrennt verwaltet.
\subsubsection{Zimmer}
Das Hotelzimmer ist ein Raum den ein Kunde gegen Bezahlung in Anspruch nehmen kann.
\subsubsection{Zimmerstatus}
Der Zimmerstatus beschreibt in welcher Lage sich das Zimmer im aktuellen Moment befindet.
\subsubsection{Kunde}
Der Kunde ist die Instanz die in direkten Zusammenhang mit einer Rechnung steht.
Einem Kunden ist eindeutig eine Rechnung zuzuordnen.
\subsubsection{Rechnung}
Eine Rechnung wird erstellt und durch ein Zahlungsmittel beglichen, dann gilt die Rechnung als abgeschlossen. Eine Rechnung hat folgende Bestandteile.

\newpage

\section{Dynamisches Modell}
\subsection{Detaillierte Benutzungsfallbeschreibungen}
\input UseCases/checkin_walkin
\input UseCases/checkin_reisegruppe
\input UseCases/checkin_reservierung
\input UseCases/checkout
\input UseCases/reservierung_firmen
\input UseCases/reservierung_individualgast
\input UseCases/reservierung_aendern
\input UseCases/reservierung_bestaetigen
\input UseCases/reservierung_stornieren_individualperson
\input UseCases/reservierung_stornieren_vertragspartner
\input UseCases/aufenthalt_verlaengern
\input UseCases/zimmer_wechsel

\newpage

\section{Nichtfunktionale Anforderungen}
\subsection{Regeln}
\subsection{Usability}
Natürlich wird versucht die Software selbsterklärend zu gestalten, trotzdem inkludiert das Projekt eine fortlaufende Dokumentation, sowie eine Hilfe, die die Einarbeitungszeit wesentlich verringern soll.
\subsection{Zuverlässigkeit}
Die Datenbank, auf der das System arbeitet, sollte sich immer einen konsistenten Zustand merken, damit im Fehlerfall ein einfacher Neustart des Systems zur Behebung ausreicht. Trotzdem sollten am besten noch automatisiert Backups erstellt werden, um die Wahrscheinlichkeit eines zu großen Datenverlusts zu begrenzen.
\subsection{Performanz}
Es ist für die Zufriedenheit des Kunden, und somit auch für den \Gls{Rezeptionist}, äußerst wichtig dass oft vorkommende Tätigkeiten schnell abgehandelt werden können. Dazu zählen vor allem der \Gls{Checkin}, und der \Gls{Checkout}, welche, insofern keine Daten nachzutragen bzw. andere Zusatzaufwände zu erledigen sind, nicht länger als 2 Minuten dauern sollten.

Außerdem ist die ständige Verfügbarkeit des Systems von großer Bedeutung, damit Anfragen von Kunden bzw. Gästen sofort bearbeitet werden können. Sollte die Verfügbarkeit für einen gewissen Zeitraum nicht gegeben sein, z.B. wegen Wartungsarbeiten, dann muss dieser Ausfall bereits im Vorherein kommuniziert werden, sodass das Persoal sich auf diese unangenehme Situation vorbereiten kann.
\subsection{Unterstützbarkeit}
\subsection{Schnittstellen}
\subsubsection{Benutzerschnittstellen}
Die Software wird vom Benutzer über eine grafische Benutzeroberfläche auf handelsüblichen Desktoprechnern bedient.
\subsubsection{Software Schnittstellen}
Es sind Schnittstellen zu den bestehenden Systemen (Finanzbuchhaltung, Debitorenbuchhaltung, Food and Beverage Verwaltung) vorzusehen.
\subsubsection{Kommunikationsschnittstellen}
Es ist nicht vorgesehen, dass das System von jedem Mitarbeiter des Hotels zu bedienen ist, da es gerade bei Mitarbeiter in Positionen mit hoher Fluktuation zu teuer wäre, jeden einzelnen dem Umgang mit der Software zu erläutern. Daher werden immer noch diverse Formulare existieren, die von gewissen Berufsgruppen (z.B. Reinigungspersonal) ausgefüllt, und danach von Mitarbeitern im Backoffice manuell ins System übertragen werden.

\newpage

\section{Iterationsplan (Timeboxes)}

\newpage

%Glossar
\addcontentsline{toc}{section}{Glossar}
\printglossary[title=Glossar,toctitle=GLOSSAR]
\end{document}
