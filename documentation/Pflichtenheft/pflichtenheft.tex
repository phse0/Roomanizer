\documentclass[10pt,a4paper,titlepage]{article}
%packages to use
\usepackage[utf8]{inputenc}
\usepackage[ngerman]{babel}
\usepackage{fancyhdr}
\usepackage{hyperref}
\usepackage[xindy]{glossaries}
\usepackage[T1]{fontenc}
\usepackage{enumitem}

%preamble
\newlist{longenum}{enumerate}{6}
\setlist[longenum,1]{label=\arabic*.}
\setlist[longenum,2]{label=\alph*)}
\setlist[longenum,3]{label=\arabic*.}
\setlist[longenum,4]{label=\alph*)}
\setlist[longenum,5]{label=\arabic*.}
\setlist[longenum,6]{label=\alph*)}
\pagestyle{fancy}
\makeglossaries

%glossary entries
\newglossaryentry{Gast}{name=Gast, plural=Gäste, description={Eine Person, die die Leistungen des Hotels in Anspruch nimmt}}
\newglossaryentry{Rezeptionist}{name=Rezeptionist, description={Ein Mitarbeiter des Hotels, welcher Personen im Eingangsbereich empfängt, und deren Anfragen bearbeitet}}
\newglossaryentry{Reservierung}{name=Reservierung, description={Eine Vormerkung für die Nutzung eines Zimmer für einen gewissen Zeitabschnitt in der Zukunft}}
\newglossaryentry{Zimmer}{name=Zimmer, description={In unserem System ist ein Zimmer ein Raum, welcher als Nächtigungsmöglichkeit für Gäste eingerichtet ist}}
\newglossaryentry{Reservierungsnummer}{name=Reservierungsnummer, description={Eine Nummer zur eindeutigen Identifikation einer Reservierung}}
\newglossaryentry{Checkin}{name=Checkin, description={Der Prozess der abgearbeitet wird, wenn ein Gast an die Rezeption gelangt}}
\newglossaryentry{Rezeption}{name=Rezeption, description={Die Stelle an der Personen empfangen werden, meistens im Eingangsbereich zu finden}}
\newglossaryentry{Zusatzleistung}{name=Zusatzleistung, description={Eine Leistung die standardmäßig nicht inkludiert ist, und über ein zusätzliches Entgelt in Anspruch genommen werden kann}}
\newglossaryentry{Belegungsnummer}{name=Belegungsnummer, description={Eine Nummer, die zur Identifikation von Rechnungen einzelner Gäste auf dem gleichen Zimmer dient.}}
\newglossaryentry{Reiseleiter}{name=Reiseleiter, description={Die Person, die für eine Reisegruppe verantwortlich ist, und auf dessen Namen die Reservierung lautet}}
\newglossaryentry{Zimmernummer}{name=Zimmernummer, description={Eine eindeutige Identifikation für ein Zimmer in einem Hotel}}
\newglossaryentry{Zwischenrechnung}{name=Zwischenrechnung, description={Eine Aufstellung aller bis zu diesem Zeitpunkt angefallenen Konsumationen}}
\newglossaryentry{Rechnung}{name=Rechnung, description={Ein Beleg für die Inanspruchnahme von gewissen Leistungen und deren Begleichung im Hotel}}
\newglossaryentry{Checkout}{name=Checkout, description={Der Prozess der abgearbeitet wenn der Gast seinen Aufenthalt im Hotel beendet}}
\newglossaryentry{Optionsdatum}{name=Optionsdatum, description={Datum, bis zu welchem die Reservierung spätestens bestätigt sein muss}}
\newglossaryentry{Belegungsvorschau}{name=Belegungsvorschau, description={Eine Vorschau des Reservierungsstandes pro Kategorie}}
\newglossaryentry{Vertragspartner}{name=Vertragspartner, description={Eine Firma, welche einen Vertrag mit dem Hotel hat und spezielle Konditionen ausgehandelt hat}}

%Real Document
\begin{document}

%Footer/Header Definitions
\fancyhf{}

\lhead{\leftmark}
\rhead{Seite \thepage}

%Some Options
\setcounter{secnumdepth}{5}
\setcounter{tocdepth}{3}

%Title page
\title{Pflichtenheft Roomanizer}
\author{Ramon Lopez \and Rafael Neumann \and Daniel Rotter \and Andreas Sinz \and Philipp Steiner}
\date{\today}
\maketitle

%Content
\tableofcontents
\newpage

\section{Einführung}

\newpage

\section{Stakeholder- und Benutzerbeschreibung}

\subsection{Stakeholder\slash{}Benutzer Zusammenfassung}
\begin{tabular}{|l|p{4cm}|p{4cm}|}
    \hline
    \textbf{Name} & \textbf{Rolle\slash{}Funktion} & \textbf{Interessiert an} \\
    \hline
    Geschäftsfuehrer & 
Vertritt die Gesellschaft gerichtlich und ausergerichtlich, leitet die Geschäfte des Hotels und hat gegenueber Dritten eine unbeschränkte Vertretungsmacht, auch repräsentative Tätigkeiten &
    Verschiedene Möglichkeiten zur Kontrolle über die aktuelle Situation des Hotels, z.B. durch Bereitstellung von Reports.\\
    \hline
    Manager &
	Ist für den finaziellen Erfolg verantwortlich und koordiniert das Personal &
	Fehlerfreies System, mit welchem man ganz einfach und schnell Reservierungen, Buchungen usw. verwalten kann. \\
    \hline
    Rezeptionist &
	Empfängt den Gast an der Rezeption kümmert sich um Checkin, Check-out und um die Probleme, Wünsche, Anregungen der Gäste. &
	Schnelles, zuverlässiges System um Kundenwünsche zu erfüllen. Der Fokus während des Gesprächs mit dem Kunden sollte aber so wenig wie möglich am Bildschirm liegen. \\
    \hline
    Koch &
Personeneinsatzplanung, Speisekartengestalten, Anleitung der Mitarbeiter der Küche und Wareneinkauf &
	System mit zuverlässigen Benachrichtigungen an seine Mitarbeiter für die Einsatzplanung, wünschenswert mit Kalendereinteilung und aktueller Warenstand für die Küche. \\
    \hline
    Zimmermädchen & 
	Reinigt die Zimmer, füllt die Minibar auf. &
	Sie erledigt Ihre Aufgabe möglichst schnell, um die Gäste nicht zu stören. \\
    \hline
    Tourismuskauffrau &
	Arbeitet an der Rezeption, Reservierung und Check-In, Organisation von Veranstaltungen für die Gaeste, Einteilung der Zimmer falls nötig &
	Leicht mit Tastatur bedienbares System welches leicht zu Erlernende Tastenkürzeln  \\
    \hline
    Gast &
	Konsumiert die Dienstleistungen des Hotels. Ist geschäftlich Unterwegs &
	Im Hotel übernachten um am nächsten Tag an einem Kongress teilzunehmen. \\
    \hline
    Auftraggeber &
	Er kennt die Anforderungen im Detail. Und stellt stellt sein Wissen dem Entwickler zur Verfügung. Er entscheidet über die Umsetzung. &
	Erfolgreiche Umsetzung und Integration des Produkts. Und die Kosten dabei im Rahmen halten. \\
    \hline
    Entwickler &
Setzt die Angaben des Auftraggebers in ein Softwareprodukt um.&
Zufriedenheit des Auftraggebers bezüglich dem erstellten Produkt.\\
    \hline
\end{tabular}

\newpage

\section{Produkt Überblick}

\newpage

\section{Domänenmodell}

\newpage

\section{Dynamisches Modell}
\subsection{Detaillierte Benutzungsfallbeschreibungen}
\input UseCases/checkin_reisegruppe
\input UseCases/checkin_reservierung
\input UseCases/checkout
\input UseCases/reservierung_firmen
\input UseCases/reservierung_individualgast
\input UseCases/reservierung_aendern
\input UseCases/reservierung_bestaetigen
\input UseCases/reservierung_stornieren_individualperson
\input UseCases/reservierung_stornieren_vertragspartner
\input UseCases/aufenthalt_verlaengern
\input UseCases/zimmer_wechsel

\newpage

\section{Nonfunktionale Anforderungen}

\newpage

\section{Iterationsplan (Timeboxes)}

\newpage

%Glossar
\addcontentsline{toc}{section}{Glossar}
\printglossary[title=Glossar,toctitle=GLOSSAR]
\end{document}
